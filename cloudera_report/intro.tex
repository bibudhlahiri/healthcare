%---------------------
\section{Introduction}
\label{sec:introduction}
%---------------------
Healthcare expenditure in the US is growing concern. The total annual healthcare expenditure in 2012 was \$2.8 trillion, the prescription drug spending accounting for \$260.8 billion \cite{healthaffairs}. As the percentage of GDP (17.9), it is the highest among all nations. The healthcare cost per capita in 2010 was \$8,233, which was one-fifth of the personal income per capita (\$42,693). However, the huge expenditure does not necessarily buy Americans a better healthcare: in 2010, the number of practicing physicians per 1,000 people was only 2.4; whereas the corresponding number for the OECD countries was 3.1; there were 2.6 hospital beds per 1,000 people in 2009, whereas the OECD average was 3.4. Moreover, a major cause of the huge expense is unnecessary and inefficient measures, which include, but are not limited to: redundant medical tests \cite{seattle}, high cost of patented prescription drugs, infections caught from hospitals and frequent readmissions to hospitals. In 2011, the unnecessary expenditures added up to \$476 billion (18\%) to \$992 billion (37\%) of total (2.6 trillion) \cite{unnecessary}.\\

We attempt to address the problem of rise in healthcare expenditure by asking the following question: what factors increase an individual's expenses on healthcare? Is it because people develop certain chronic conditions? Is it because of age? Is it due to side-effects of drugs they have been prescribed? More specifically, given an individual's demographic and medical information, how well can we predict whether or not the individual's healthcare expense will rise next year? These questions can be of interest to the patients/beneficiaries, the insurance providers and the healthcare service providers in ways we explain below:
\begin{itemize}
\item {\bf Beneficiaries:} If the beneficiaries know in advance that they are under high risk of an increase in expenditure for the next year, they can choose the insurance plans with higher deductible with more confidence. That way, although the annual out-of-pocket expense until the deductible is met would be higher, the beneficiary can benefit from a low monthly premium, and has to pay only 20-30\% co-insurance out-of-pocket once the deductible is met. 
\item {\bf Insurance providers:} The insurance providers can project the cost and revenue for the next year more accurately, given the data about the beneficiaries registered with them.
\item {\bf Healthcare service providers:} Hospital admissions and stays are intrinsically expensive. If doctors and hospitals know which patients are the high-risk ones, they can avoid readmission by taking preventive measures, e.g., arranging more frequent check-ups as outpatients, making the patients wear inexpensive sensors when they are discharged to monitor their conditions, etc.
\end{itemize}

{\bf Dataset: } We worked on the dataset ``CMS 2008-2010 Data Entrepreneurs$^\prime$ Synthetic Public Use File (DE-SynPUF)'' available at \cite{desynpuf}. This is an anonymized dataset obtained from Medicare \cite{medicare}, and has data about 6.87 million beneficiaries, their inpatient claims, outpatient claims and prescription drug events. Since inpatient claims often form a significant fraction of an individual's annual healthcare expenses, we analyzed the expenditure on the inpatient claims. We use the other data within the dataset to derive features based on the prescription drugs taken by the beneficiaries, the medical conditions they got diagnosed with as inpatients or even as outpatients, and the chronic conditions they had. One problem of healthcare datasets, in general, is that different aspects of a person's health are often siloed in different databases. This dataset is free from that problem. The dataset is anonymized to protect the privacy of the beneficiaries, but the various parts of the dataset are linked through beneficiary IDs that were created by applying a hash function on the original beneficiary IDs. We present more details of the dataset in Section~\ref{sec:data}.\\

{\bf Contributions:} Our contribution can be summarized as follows:
\begin{itemize}
\item We used a large, publicly available dataset with more than 114,000 beneficiaries and their health history for a span of three years (2008-2010) to investigate which ones, among more than 12,400 features, led to increase in expenditure as inpatient between 2008 and 2009. We formulated it as a binary classification problem, and brought down the number of features from 12,400 to 44 through feature selection. 
\item We experimented with various classification algorithms and finally chose six (gradient boosting machine \cite{Friedman01}, conditional inference tree \cite{HHZ06}, neural networks \cite{Ripley96}, SVM \cite{CV95}, logistic regression and naive Bayes) that performed best on a held-out test data set, and finally applied stacked generalization \cite{Wolpert92} as the ensemble technique. This achieved a sensitivity (recall) of 80\%, an overall accuracy of 77.56\% and a precision of 76.46\% on the test data set. This way, we showed that it is indeed possible to develop learning models which will predict with great degree of accuracy about whether an individual is going to incur higher or lower healthcare expenditure based on normally collected information.
\item Lastly, we identified major factors which are crucial in determining whether an individual is going to incur higher healthcare expenditure going forward. 
\end{itemize}

 
