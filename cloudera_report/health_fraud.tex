\documentclass{sig-alternate}

\usepackage{amsmath}
\usepackage{booktabs}
\usepackage{pgfplots}
\usepackage{pgfplotstable}

\usepackage{hyperref}
%\usepackage{underscore}
\usepackage[vertfit]{breakurl}

\newtheorem{lemma}{Lemma}[section]
\newtheorem{fact}{Fact}[section]
\newtheorem{theorem}{Theorem}[section]
\newtheorem{question}{Question}[section]
\newtheorem{constraint}{Constraint}[section]
\newtheorem{definition}{Definition}[section]
\newtheorem{problem}{Problem}
\newcommand{\remove}[1]{}

\usepackage{graphicx,epsfig,subfig,float}
\usepackage[algoruled,linesnumbered]{algorithm2e}

\newcommand{\qedsymb}{\hfill{\rule{2mm}{2mm}}}
%\newenvironment{proof}{\begin{trivlist}
%\item[\hspace{\labelsep}{\bf\noindent Proof: }]}{\qedsymb\\\end{trivlist}}

\begin{document}

\title{Medicare Claim Anomaly Detection Challenge}

\numberofauthors{2}
\author{
\alignauthor
Bibudh Lahiri\\
       \affaddr{Impetus Technologies}\\
       \affaddr{720 University Avenue}\\
       \affaddr{Los Gatos, CA 95032}\\
       \email{blahiri@impetus.com}
}

\maketitle
\begin{abstract}
Healthcare expenditure is a growing concern in the US. In 2012, the total US annual healthcare expenditure reached \$2.8 trillion, and according to the National Health Care Anti-Fraud Association (NHCAA), the financial losses rising from healthcare fraud amounts to tens of billions of dollars each year. There are various ways fraud is committed in the health industry, some of the most common ones being billing for services that were never rendered to a patient, or billing for more expensive services or procedures than were actually provided or performed (``upcoding''). In this challenge organized by Cloudera, we worked with a Medicare claims dataset, that included data about amounts claimed by providers for doing different procedures on patients registered with Medicare and the amounts reimbursed by Medicare; to answer specific and open-ended questions that potentially indicate malpractice conducted by some providers. We found that providers in some very specific geographic regions appearing repeatedly in different parts of the analysis, thereby demanding close scrutiny. 
\end{abstract}

\keywords{Healthcare, expenditure, fraud detection}

%%---------------------
\section{Introduction}
\label{sec:introduction}
%---------------------
Healthcare expenditure in the US is growing concern. The total annual healthcare expenditure in 2012 was \$2.8 trillion, the prescription drug spending accounting for \$260.8 billion \cite{healthaffairs}. As the percentage of GDP (17.9), it is the highest among all nations. The healthcare cost per capita in 2010 was \$8,233, which was one-fifth of the personal income per capita (\$42,693). The health expenditure per capita was about \$3,000 more than Norway, Switzerland or Netherlands, nations who came closest \cite{countrycomp} in terms of healthcare cost per capita. However, the huge expenditure does not necessarily buy Americans a better healthcare: in 2010, the number of practicing physicians per 1,000 people was only 2.4: whereas the corresponding number for the OECD countries was 3.1; there were 2.6 hospital beds per 1,000 people in 2009, whereas the OECD average was 3.4. In terms of life expectancy, the US ranked $28^{th}$ among the OECD countries in 2008, and it ranked below the average \cite{lifeex}. Moreover, a lot of the huge expense in the US is being attributed to unnecessary and inefficient measures, which include, but are not limited to: redundant medical tests \cite{seattle}, high cost of patented prescription drugs, infections caught from hospitals and frequent readmissions to hospitals. In 2011, the unnecessary expenditures added up to \$476 billion (18\%) to \$992 billion (37\%) of total (2.6 trillion) \cite{unnecessary}.\\

Policymakers and practitioners have started taking initiatives to reduce the overall expenditure and to tame its growth. Some of the measures suggested (and taken) are to increase the use of generic drugs as opposed to patented brand-name drugs, to identify services that were once considered good healthcare but now are suspected to lack evidence of benefit, to reduce the time spent due to administrative complexities, etc. Awareness about the usage of health data to derive useful insights for the patients, as well as for the service providers, is also on rise. As part of this initiative, the    
Department of Health and Human Services (HHS) took the Health Data Initiative \cite{hdi}, as a result of which, many health-related datasets, some of which are available from the insurance providers, are now publicly available on sites like \cite{healthdata}.\\

We attempt to address the problem of rise in healthcare expenditure by asking the following question: what factors increase an individual's expenses on healthcare? Is it because people develop certain chronic conditions? Is it because of age? Is it due to side-effects of drugs they have been prescribed? More specifically, given an individual's demographic and medical information, how well can we predict whether or not the individual's healthcare expense will rise next year? Although we formulated the problem as a binary classification problem, and experimented with various machine learning techniques to improve the classification accuracy, our goal was to go beyond simply achieving a high predictive accuracy: to identify how important the different factors are in increasing a person's healthcare expenditure.\\

Besides helping an individual, this can also be of interest to the patients/beneficiaries, the insurance providers and the healthcare service providers in ways we explain below:
\begin{itemize}
\item {\bf Beneficiaries:} If the beneficiaries know in advance that they are under high risk of an increase in expenditure for the next year, they can choose the insurance plans with higher deductible with more confidence. That way, although the annual out-of-pocket expense until the deductible is met would be higher, the beneficiary can benefit from a low monthly premium, and has to pay only 20-30\% co-insurance out-of-pocket once the deductible is met. 
\item {\bf Insurance providers:} The insurance providers can project the cost and revenue for the next year more accurately, given the data about the beneficiaries registered with them.
\item {\bf Healthcare service providers:} Hospital admissions and stays are intrinsically expensive. If doctors and hospitals know which patients are the high-risk ones, they can avoid readmission by taking preventive measures, e.g., arranging more frequent check-ups as outpatients, making the patients wear inexpensive sensors when they are discharged to monitor their conditions, etc.
\end{itemize}

{\bf Dataset: } We worked on the dataset ``CMS 2008-2010 Data Entrepreneurs$^\prime$ Synthetic Public Use File (DE-SynPUF)'' available at \cite{desynpuf}. This is an anonymized dataset obtained from Medicare \cite{medicare}, and has data about 6.87 million beneficiaries, their inpatient claims (claims filed on hospitalization), outpatient claims and prescription drug events. Since inpatient claims often form a significant fraction of an individual's annual healthcare expenses, we analyzed the expenditure on the inpatient claims. However, we use the other datasets (as well as some auxiliary datasets) to derive features on which prescribed drugs the beneficiaries took, which medical conditions they got diagnosed with as inpatients or even as outpatients, and what chronic conditions they had. One problem of healthcare datasets, in general, is that different aspects of a person's health are often siloed in different databases. This dataset is free from that problem, and that was one of the main reasons we chose it. 
The data is anonymized to protect the privacy of the beneficiaries, but the various parts of the dataset are linked through beneficiary IDs that were created by applying a hash function on the original beneficiary IDs. We present more details of the dataset in Section~\ref{sec:data}.

{\bf Contributions:} Our contribution can be summarized as follows:
\begin{itemize}
\item We used a large, publicly available dataset with 6.87 million beneficiaries and their comprehensive health history for a span of three years (2008-2010) to investigate what factors led to increase in expenditure as inpatient between 2008 and 2009. We formulated it as a binary classification problem. However, since the positive class (beneficiaries whose inpatient expenditure increased between 2008 and 2009) constituted only 14\% of the original dataset we started with, applying binary classification algorithms on that data itself (or a simple random sample of it) would have resulted in high false negative rate (low sensitivity), even if it had achieved high overall accuracy - the so-called ``class imbalance'' problem \cite{FP00}. To avoid this, we oversampled the minority class and undersampled the majority class to create a sample with equal representation of both classes.  
\item We experimented with various classification algorithms and finally chose six (SVM \cite{CV95}, gradient boosting machine \cite{Friedman01}, conditional inference tree \cite{HHZ06}, logistic regression, naive Bayes and neural networks \cite{Ripley96}) that performed best on a held-out test data set, and finally applied stacked generalization \cite{Wolpert92} as the ensemble technique. This achieved a sensitivity (recall) of 80\%, an overall accuracy of 77.56\% and a precision of 76.46\% on the test data set. 
\end{itemize}

 

%%---------------------
\section{Details on Data}
\label{sec:data}
%---------------------
As we mentioned in Section~\ref{sec:introduction}, the DE-SynPUF dataset \cite{desynpuf} has information about 6.87 million beneficiaries, their inpatient claims (claims filed on hospitalization), outpatient claims and prescription drug events. Each of these components is available in 20 different partitions, and we worked with the first partition (about 5\% of the data). We created a PostgreSQL \cite{postgresql} database with this partition, and used \texttt{RPostgreSQL} \cite{rpostgresql} to connect to the database. The volumes of the different parts of the dataset are in Table~\ref{tab:desynpuf_data}.\\

\begin{table}[ht]
\caption{Volume of DE-SynPUF dataset}
\begin{tabular}{lrr}
\hline
Entities & Whole data & Subset explored\\
\hline
Beneficiaries & 6.87 million & 116k\\
Inpatient claims & 1.3 million & 66.7k\\
Outpatient claims & 15.8 million & 790.8k\\
Prescription drug events & 111 million & 5.5 million\\
\hline
\end{tabular}
\label{tab:desynpuf_data}
\end{table}

We now present some details of each of the components of the dataset (a mode detailed data dictionary is availble at \cite{datadic}): 
\begin{enumerate}
\item {\bf Beneficiaries:} We worked with a subset of 114,538 beneficiaries who were registered in both 2008 and 2009 with Medicare. The beneficiary subset provides the gender, date of birth, gender, race, whether the beneficiary had end stage renal disease, whether the beneficiary had any of the following chronic conditions: Alzheimer or related disorders or senility, heart failure, kidney disease, cancer, chronic obstructive pulmonary disease (COPD), depression, diabetes, ischemic heart disease, osteoporosis, rheumatoid arthritis and osteoarthritis, stroke/transient ischemic attack. The beneficiary summary is stored at a per-year level. That helped us to derive features like whether a beneficiary developed a chronic condition in 2009 which she did not have in 2008, and as we will show later, those derived features turned out to be pretty strong factors behind cost increase. The beneficiary dataset also had information about the total amount that Medicare reimbursed in a year for inpatient and outpatient claims by a beneficiary.  
\item {\bf Inpatient claims:} We worked with a subset of 66.7k inpatient claims, which had, for each claim, the admission and discharge dates for hospitalization episodes, and a list of three types of codes:
\begin{enumerate}
\item {\bf Diagnosis codes: }These are ICD9 \cite{icd9} codes for beneficiary's principal or other diagnosis. They capture the physician's opinion of the patient's specific illnesses, signs, symptoms, and complaints. For example, 414.00 is the ICD9 code for Coronary Atherosclerosis.
\item {\bf Procedure codes: }These are ICD9 codes for specific health interventions made by medical professionals, e.g., 4513 stands for ``Other endoscopy of small intestine''.
\item {\bf HCPCS codes: }These are CPT codes \cite{cpt} for tasks and services a medical practitioner may provide to a Medicare patient including medical, surgical and diagnostic services, e.g., 90658 stands for ``flu shot''.
\end{enumerate}
\item {\bf Outpatient claims:} We worked with a subset of 791k outpatient claims, which had, for each claim, a list of diagnosis codes, procedure codes and HCPCS codes, like inpatient claims.
\item {\bf Prescription Drug Events: }We worked with a subset of 5.5 million prescription drug events, where each record had a product service ID, which identifies the dispensed drug using a National Drug Code (NDC); the number of units, grams or milliliters dispensed and the number of days' supply of medication.
\end{enumerate}

We complemented this dataset by auxiliary ones that give: 
\begin{itemize}
\item The short and long descriptions of diagnoses for given ICD9 codes
\item The short and long descriptions of procedures for given ICD9 codes
\item A dataset that, given the NDC code of a drug, gives the name of the substance (the chemical ingredient), the product type (``human OTC drug'', ``human prescription drug'' etc), the non-proprietary name, routename (intramuscular, intravenous, oral etc), labeler name etc. For example, NDC code ``0003-0855'' stands for a drug named Sprycel \cite{sprycel}, that is used for treating a type of leukemia (CML), has to be taken orally, has a non-proprietary name Dasatinib, which also is the name of the substance behind. As we will discuss later, we actually used the substance names instead of the NDC codes when we did the modeling, since many drugs with different NDC codes and different proprietary names can have the same substance behind.  
\end{itemize}
 
Before we started working on the classification problem, we did some exploratory analysis of the DE-SynPUF dataset, and we present some of its results now: we had a total of 114,538 beneficiaries who were registered in both 2008 and 2009 with Medicare. 55\% of them were female, 45\% male. The median age at the start of 2009 was 72. We show the distribution of age (histogram and kernel density estimate) at the beginning of 2009 in Figure~\ref{fig:age_2009}. We notice a spike at the age 65, since Medicare is offered to people with age 65 or more, and with younger people with disabilities or end stage renal disease (ESRD). The median number of inpatient claims, among people who did get hospitalized, was one, for both 2008 and 2009. A vast majority of inpatient claim amounts were 0, implying most of the registered beneficiaries never got hospitalized. However, some non-zero values of inpatient claim amounts were very large. In Figures~\ref{fig:ip_claim_amt_2008_distn} and ~\ref{fig:ip_claim_amt_2009_distn}, we show the distribution of inpatient claim amounts for 2008 and 2009, respectively. We see that although 75-80\% of the patients had an inpatient claim amounts between \$0 and \$100, the fraction of patients with inpatient claim amounts \$5,000 and above is not very small. The mean amounts of inpatient claims in these two years are \$2,583 and \$2,526 respectively. Overall, we see an increase in inpatient claim amounts for 16,248 (14.2\%) beneficiaries, while for the remaining 98,290 (85.8\%) beneficiaries, it remained same or did not increase.


\begin{figure}[!h]
    \centering
    \includegraphics[width=0.4\textwidth]{/Users/blahiri/healthcare/code/figures/age_2009_distn.png}
    \caption{\small Age distribution at the beginning of 2009}
    \label{fig:age_2009}
\end{figure}

\begin{figure}[!h]
\centering
\subfloat[Inpatient claim amounts in 2008]{\includegraphics[width=0.4\textwidth]{/Users/blahiri/healthcare/code/figures/ip_claim_amt_2008_distn.png} \label{fig:ip_claim_amt_2008_distn}}\\
\subfloat[Inpatient claim amounts in 2009]{\includegraphics[width=0.4\textwidth]{/Users/blahiri/healthcare/code/figures/ip_claim_amt_2009_distn.png} \label{fig:ip_claim_amt_2009_distn}}
\caption{\small Inpatient claim amounts, showing the overall pattern did not change much between 2008 and 2009}
\end{figure}

%---------------------
\section{Basic Analysis}
\label{sec:ppr}
%---------------------
The first problem demanded investigating into some very specific questions. There may be providers and regions from where bills for conducting certain procedures on patients are systematically higher compared to other regions and procedures. We describe the questions and provide our solutions in the following subsections:

%---------------------
\subsection{Procedures with highest relative variance in cost}
\label{subsec:highest_rel_var}
%---------------------
The goal was to identify the three procedures with the highest relative variance in cost. The procedures can be conducted as in-patients or out-patients: they were in two separate files: \texttt{Medicare\_Provider\_Charge\_Inpatient\_DRG100\_FY2011.csv} and \texttt{Medicare\_Provider\_Charge\_Outpatient\_APC30\_CY2011\_v2.csv}, respectively. Both had the data aggregated at the level of (provider, procedure, average charge), so each row gave us how much a provider charged for carrying out a procedure on average, where the average was taken over all patients who underwent that procedure in 2011.  We loaded these two datasets to two separate tables in Hive (we used Hive 0.10.0 on Cloudera Hadoop), respectively \texttt{provider\_charge\_inpatient} and \texttt{provider\_charge\_outpatient}, and combined them into one (\texttt{prov\_proc\_charge}) using the \texttt{union all} feature of Hive, and then computed mean ($\bar{x}$) and variance ($s^2$) for each procedure by the \texttt{group by}, \texttt{var\_pop} and \texttt{avg} functions/features of Hive. The query to compute mean, variance and relative variance is in Algorithm~\ref{algo:highest_rel_var}.\\

\begin{algorithm}
\vspace{-2pt}
\caption{{\sf \texttt{highest\_rel\_var}}()}
\label{algo:highest_rel_var}
\begin{verbatim}
create table relative_variance(
  procedure_def STRING,
  mean_submitted_charge DOUBLE,
  variance_of_submitted_charge DOUBLE,
  rel_var DOUBLE
);
insert into table relative_variance
select a.procedure_def, a.mean_submitted_charge, 
       a.variance_of_submitted_charge, 
       a.variance_of_submitted_charge/a.mean_submitted_charge
from (select procedure_def, 
      avg(avg_charge) as mean_submitted_charge, 
      var_pop(avg_charge) as variance_of_submitted_charge
      from prov_proc_charge
      group by procedure_def) a;
\end{verbatim}
\end{algorithm}


We used the following definition of relative variance: if $\bar{x}$ is the average charge for a procedure (averaged across all providers), and $s$ is the standard deviation, then the relative variance is $\frac{s^2}{\bar{x}}$. Using this definition, the three procedures with the highest relative variance were: 870 (SEPTICEMIA OR SEVERE SEPSIS W MV 96+ HOURS), 207 ( RESPIRATORY SYSTEM DIAGNOSIS W VENTILATOR SUPPORT 96+ HOURS) and 853 (INFECTIOUS \& PARASITIC DISEASES W O.R. PROCEDURE W MCC). Note that although we combined the inpatient and outpatient data to arrive at these results, the ones with the highest relative variance were all inpatient procedures. The result is in the attached file part1a.csv.


%---------------------
\subsection{Providers with highest claim amounts}
\label{subsec:providers_with_most_max}
%---------------------
This problem was to identify the three providers who claimed the highest amounts for the largest number of procedures. Similar to the problem in Section\ref{subsec:highest_rel_var}, we used the table \texttt{prov\_proc\_charge} with combined inpatient and outpatient data, and then did the remaining task by Algorithm~\ref{algo:providers_with_most_max}. The intermediate table \texttt{highest\_avg} contains the highest charge for each procedure; and in the next step it is joined with \texttt{prov\_proc\_charge} to retrieve the provider who claims the highest charge for a given procedure. Once that is stored in \texttt{providers\_with\_highest\_avg}, the third query retrieves the number of procedures in which a provider turns out to be the most expensive one. The providers who turn out to be the most expensive ones in the maximum number of procedures, as given in the attached file part1b.csv, are 310025 (Bayonne Hospital Center in Bayonne, NJ), 390180 (Crozer Chester Medical Center in Upland, PA) and 50441 (Stanford Hospital in Stanford, CA). Note that although the providers in the given data are spread all over the US, these three are all from the northeastern region and California - so it may even be because of the generally high cost of living in these areas. 

\begin{algorithm}
\vspace{-2pt}
\caption{{\sf \texttt{providers\_with\_most\_max}}()}
\label{algo:providers_with_most_max}
\begin{verbatim}
create table highest_avg as
select procedure_def, max(avg_charge) as highest_charge
from prov_proc_charge
group by procedure_def;

create table providers_with_highest_avg as
select ppc.procedure_def, 
       ppc.provider_id as most_exp_prov, 
       ha.highest_charge
from prov_proc_charge ppc join highest_avg ha 
on (ppc.procedure_def = ha.procedure_def 
    and ppc.avg_charge = ha.highest_charge);

create table providers_with_max_highest as
select a.most_exp_prov as provider, 
       count(distinct a.procedure_def) 
          tops_in_procedures
from providers_with_highest_avg a
group by a.most_exp_prov;
\end{verbatim}
\end{algorithm}

%---------------------
\subsection{Regions with largest average amount}
\label{subsec:regions_with_most_max}
%--------------------
The goal was to identify the three regions in which the providers claimed the highest average amount (averaged across all providers in the region) for the largest number of procedures. Similar to the problem in Section\ref{subsec:highest_rel_var}, we used the table \texttt{prov\_proc\_charge} with combined inpatient and outpatient data, and then did the remaining task by Algorithm~\ref{algo:regions_with_most_max}. The first query computes the mean charge for each combination of region and procedure (mean computed across all providers in the region), and stores it in \texttt{region\_proc\_charge} . The second query, which takes advantage of the window function feature of PostgreSQL, finds out the region for each procedure where the average charge for conducting the procedure is highest; and stores it in \texttt{regions\_with\_highest\_avg}. The third query computes, for each region, the number of procedures for which providers in that region turn out to be the most expensive ones, and stores it in \texttt{regions\_with\_most\_max}. The result is in the attached file part1c.csv. The top three are: CA - Contra Costa County, CA - San Mateo County and CA - Santa Cruz, and these regions are most expensive in terms of 36, 24 and 11 procedures respectively. Note that Stanford Hospital, mentioned in Subsection~\ref{subsec:providers_with_most_max} as one of the providers with the highest amounts for the largest number of procedures, is in CA - San Mateo County.

\begin{algorithm}
\vspace{-2pt}
\caption{{\sf \texttt{regions\_with\_most\_max}}()}
\label{algo:regions_with_most_max}
\begin{verbatim}
drop table if exists region_proc_charge;
create table region_proc_charge as
select hrr, procedure_def, avg(avg_charge) as mean_charge
from prov_proc_charge
group by hrr, procedure_def;

drop table if exists regions_with_highest_avg;
create table regions_with_highest_avg as
select a.hrr as most_exp_region, a.procedure_def, 
       a.max_charge
from (select hrr, procedure_def, mean_charge, 
      max(mean_charge) over (partition by procedure_def) 
         as max_charge
      from region_proc_charge) a
where a.mean_charge = a.max_charge;

drop table if exists regions_with_most_max;
create table regions_with_most_max as
select a.most_exp_region as region, 
       count(distinct a.procedure_def) tops_in_procedures
from regions_with_highest_avg a
group by a.most_exp_region;
\end{verbatim}
\end{algorithm}


%---------------------
\subsection{Providers with largest claim difference}
\label{subsec:providers_with_most_claim_diff}
%--------------------
This problem was to identify the producers with the largest claim difference on the maximum number of procedures. Similar to the problem in Section\ref{subsec:highest_rel_var}, we created a table \texttt{prov\_proc\_claim\_diff} with combined inpatient and outpatient data, where we created the column \texttt{claim\_diff} as the difference between the (average) amount claimed by the provider for a procedure and the (average) amount reimbursed by Medicare. Then we did the remaining task by Algorithm~\ref{algo:providers_with_most_claim_diff}. The first query in Algorithm~\ref{algo:providers_with_most_claim_diff} creates table \texttt{highest\_claim\_diff} to store the highest claim difference for each procedure. The next query joins \texttt{prov\_proc\_claim\_diff} with \texttt{highest\_claim\_diff} to retrieve, for each procedure, the provider with the highest claim difference; and stores it in  \texttt{providers\_with\_highest\_clm\_diff}. The third query retrieves, for how many procedures a provider shows up as the one with the highest claim difference, and stores it in table \texttt{providers\_with\_max\_highest}, based on which the final result can be retrieved.\\

The three providers finally obtained were: 310025 (Bayonne Hospital Center in Bayonne, NJ), 390180 (Crozer Chester Medical Center in Upland, PA) and 390390 (Hahnemann University Hospital in Philadelphia, PA). Note that the first two of these were listed as the providers with the highest claim amount, too, in Subsection~\ref{subsec:providers_with_most_max}, so the highest claim difference may be a result of these particular providers billing high but Medicare reimbursing an amount with a low variance as per its policy. The result is in the attached file part1d.csv.

\begin{algorithm}
\vspace{-2pt}
\caption{{\sf \texttt{providers\_with\_most\_claim\_diff}}()}
\label{algo:providers_with_most_claim_diff}
\begin{verbatim}
create table highest_claim_diff as
select procedure_def, 
       max(claim_diff) as highest_clm_diff
from prov_proc_claim_diff
group by procedure_def;

create table providers_with_highest_clm_diff as
select ppcd.procedure_def, 
       ppcd.provider_id as prov_with_highest_clm_diff, 
       hcd.highest_clm_diff
from prov_proc_claim_diff ppcd join highest_claim_diff hcd 
     on (ppcd.procedure_def = hcd.procedure_def 
         and ppcd.claim_diff = hcd.highest_clm_diff);

create table providers_with_max_highest as
select a.prov_with_highest_clm_diff as provider, 
       count(distinct a.procedure_def) 
          tops_in_procedures
from providers_with_highest_clm_diff a
group by a.prov_with_highest_clm_diff;
\end{verbatim}
\end{algorithm}
 


%---------------------
\section{Outlier Providers and Regions}
\label{sec:least_likely}
%---------------------
This part of the analyis was more open-ended in terms of the questions asked, but the overall goal was to identify a set of providers and regions which are least like the others. We applied two different algorithms for providers and regions, which are described below:

%---------------------
\subsection{Outlier providers}
\label{subsec:outlier_providers}
%---------------------
We used R for this analysis as we used some advanced machine learning algorithms in the exploratory phase. We started with the inpatient and outpatient data, as in Section~\ref{sec:ppr}, and converted the data from long to wide format, i.e., to a format where the providers were along the rows (along with their names and regions), and the procedures formed the columns, so a value in a call was the average charge filed by the provider for a procedure, as given in the raw data. This made each provider a 130-dimensional point as there were 130 different procedures, and the provider-procedure matrix, with 3337 rows and 130 columns, was a sparse one. Next, we applied principal component analysis on the provider-procedure matrix and visualized the projection of all the providers on the first two principal components. The first two principal components together explained 52.87\% of variance, where the first principal component itself explained 47.03\% of variance. The resultant plot is in Figure~\ref{fig:providers_first_two_pc}. We see that while most providers belong to a central cloud, there are 15 providers for whom the projection value along the first PC (the X-axis value) is -30 or less: all these providers are in New Jersey (3 providers), Pennsylvania (2 providers) or California (10 providers). Note that providers from these regions were repeatedly appearing in the analysis reported in Section~\ref{sec:ppr}, too.\\ 

\begin{figure}[!h]
    \centering
    \includegraphics[width=0.5\textwidth]{/Users/blahiri/healthcare/code/cloudera_challenge/figures/providers_first_two_pc.png}
    \caption{\small Projection along first two PCs for providers}
    \label{fig:providers_first_two_pc}
\end{figure}  

We list the top 5 (in terms of projection along first PC) among these 15 providers in Table~\ref{tab:providers_first_two_pc}. However, this was mostly a visual test; for identifying providers who are ``least likely'' the others as per some quantitative measure, we used the following idea: we computed pairwise Eucledian distance between all pairs of providers, and then for each point, extracted its distance with its 20 nearest points. Then, for each point, we computed its mean distance with these 20 nearest neighbors, and picked the providers whose mean distance with the 20 nearest neighbors are the highest. The three providers least likely as others as per this measure were: 50195 (Washington Hospital in Fremont, CA), 50441 (Stanford Hospital in Stanford, CA) and 50367 (North Bay Medical Center in Fairfield, CA). We have listed these three in attached file part2a.csv. Note that 50441 appeared in the 3rd position in Table~\ref{tab:providers_first_two_pc}, and although we present only the top 5 rows in the table, both 50195 and 50367 also appeared among the top 15 outlier providers in the PC-based projection. Also, 50441 appeared in the result of the analysis done in Section~\ref{subsec:providers_with_most_max}.\\

To investigate what led these providers to be so unlike from the rest, we compared, for each procedure, the cost to perform it with a given outlier provider, to the average cost to perform it with all remaining providers. The plot for provider 50195 is in Figure~\ref{fig:outlier_50195_vs_rest}, where we see the cost for 50195 (green bars) is in general way higher than others (red bars), especially for some procedures like 207 (RESPIRATORY SYSTEM DIAGNOSIS W VENTILATOR SUPPORT 96+ HOURS), 329 (MAJOR SMALL and LARGE BOWEL PROCEDURES W MCC) and 853 (INFECTIOUS and PARASITIC DISEASES W O.R. PROCEDURE W MCC). Note that 207 and 853 were mentioned as procedures with highest relative variance in Subsection~\ref{subsec:highest_rel_var}.

\begin{table*}[!h]
\centering
\caption{Outlier providers based on projection on first two PCs}
\begin{tabular}{rrrll}
\hline
Provider ID & Projection along first PC & Projection along second PC  & Provider name & Region\\
\hline
390180 & -52.76278 & 2.3945784 & Crozer Chester Medical Center & PA - Philadelphia\\
50625 & -45.36287 & 10.4893065 & Cedars-Sinai Medical Center & CA - Los Angeles\\
50441 & -41.63862 & 5.8213967 & Stanford Hospital & CA - San Mateo County\\
390290 & -40.19077 & -0.2487898 & Hahnemann University Hospital & PA - Philadelphia\\
50180 & -37.25559 & 0.3429694 & John Muir Medical Center & CA - Contra Costa County\\
\hline
\end{tabular}
\label{tab:providers_first_two_pc}
\end{table*}

\begin{figure*}[!h]
    \centering
    \includegraphics[width=18cm,height=9cm]{/Users/blahiri/healthcare/code/cloudera_challenge/figures/outlier_50195_vs_rest.png}
    \caption{\small Cost to perform different procedures with provider 50195 vs others}
    \label{fig:outlier_50195_vs_rest}
\end{figure*} 

%---------------------
\subsection{Outlier regions}
\label{subsec:outlier_regions}
%---------------------
We started the analysis of the outlier regions in the same way as that of the outlier providers in Subsection~\ref{subsec:outlier_providers}. However, in the data processing phase, we had to compute the average charge, for each procedure, across all the providers in a region, so that our wide-format data became a matrix with 306 rows (one for each region) and 130 columns (one for each procedure). We then applied principal component analysis on this matrix, and plotted the projections on the first two PCs. The first two principal components together explained 65.85\% of variance, where the first principal component itself explained 61.6\% of variance. The resultant plot is in Figure~\ref{fig:regions_first_two_pc}. We see that while most regions belong to a central cloud, there are 14 regions for whom the projection value along the first PC (the X-axis value) is -20 or less: all these regions are in California (10 regions), New Jersey (2 regions), Pennsylvania (1 region) and Nevada (1 region). Note that regions in California were repeatedly appearing in the analysis reported in Section~\ref{subsec:regions_with_most_max}, too.\\ 

We list the top 8 (in terms of projection along first PC) among these 14 regions in Table~\ref{tab:regions_first_two_pc}. Note that out of the five providers listed in Table~\ref{tab:providers_first_two_pc}, one was from San Mateo County and another was from Contra Costa County in California, both of which are listed in Table~\ref{tab:regions_first_two_pc}. However, this was mostly a visual test; for identifying regions which are ``least likely'' the others as per some quantitative measure, we applied the Angle-Based Outlier Degree (ABOD) algorithm by Kriegel et al \cite{KSZ08}. It is based on a simple but powerful intuition: in the high-dimensional space the data points are in, an outlier point is supposed to be far from the central cloud of the ``normal'' points. Hence, for each point $\bf{p}$, if we take all possible pairs $(\bf{x}, \bf{y})$ from the other points, and join $\bf{p}$ with both $\bf{x}$ and $\bf{y}$ with lines, thereby creating the vectors $\bf{x} - \bf{p}$ and $\bf{y} - \bf{p}$, then the variance of the angle between $\bf{x} - \bf{p}$ and $\bf{y} - \bf{p}$, taken over all pairs $(\bf{x}, \bf{y})$, will be relatively small if $\bf{p}$ is an outlier point, and will be relatively large if $\bf{p}$ is a ``normal'' point. Since computing the angle for all possible pairs $(\bf{x}, \bf{y})$ for all points $\bf{p}$ takes $O(n^3)$ time (where $n$ is the number of points), we estimated the variance of the angles between $\bf{x} - \bf{p}$ and $\bf{y} - \bf{p}$ (this is called the ``ABOD value'') by taking a random sample of 50 pairs from the set of all possible pairs, and ranked the points $\bf{p}$ in (increasing) order of ABOD values. The three points with the lowest ABOD values, indicating the three regions which are least like the others, were CA - San Mateo County, CA - Contra Costa County and CA - San Francisco. The results are in the attached file part2b.csv. Note that all these three regions appear in Table~\ref{tab:regions_first_two_pc}, two of them appear in Table~\ref{tab:providers_first_two_pc}, and the same two of them appeared in the result presented in Subsection~\ref{subsec:regions_with_most_max}.\\

To investigate what led these regions to be so unlike from the rest, we compared, for each region, the cost to perform it in a given outlier region (averaged across all providers in that region), to the average cost to perform it in the remaining regions (averaged across all providers in the remaining regions). The plot for the region CA - San Mateo County is in Figure~\ref{fig:outlier_CA_San_Mateo_County_vs_rest}, where we see the cost for San Mateo County, CA (green bars) is in general way higher than others (red bars), especially for some procedures like 207 (RESPIRATORY SYSTEM DIAGNOSIS W VENTILATOR SUPPORT 96+ HOURS), 329 (MAJOR SMALL and LARGE BOWEL PROCEDURES W MCC), 853 (INFECTIOUS and PARASITIC DISEASES W O.R. PROCEDURE W MCC) and 870 (SEPTICEMIA OR SEVERE SEPSIS W MV 96+ HOURS). Note that 207, 870 and 853 were mentioned as procedures with highest relative variance in Subsection~\ref{subsec:highest_rel_var}.

\begin{figure}[!h]
    \centering
    \includegraphics[width=0.5\textwidth]{/Users/blahiri/healthcare/code/cloudera_challenge/figures/regions_first_two_pc.png}
    \caption{\small Projection along first two PCs for regions}
    \label{fig:regions_first_two_pc}
\end{figure}  

\begin{table*}[!h]
\centering
\caption{Outlier regions based on projection on first two PCs}
\begin{tabular}{rrrll}
\hline
Region & PC1 & PC2\\
\hline
CA - Contra Costa County & -38.44711 & -4.6195783\\
CA - San Mateo County    & -36.42923 & -0.7682284\\
CA - San Jose            & -31.27909 & 1.4654849\\
CA - Alameda County      & -27.09126 & 0.5947005\\
CA - Modesto             & -26.49986 &-3.7108748\\
NJ - New Brunswick       & -26.29346 &-0.4006588\\
CA - San Francisco       & -26.14622 & 1.4868439\\
CA - Palm Springs/Rancho & -25.55973 &-0.1272999\\
\hline
\end{tabular}
\label{tab:regions_first_two_pc}
\end{table*}

\begin{figure*}[!h]
    \centering
    \includegraphics[width=18cm,height=9cm]{/Users/blahiri/healthcare/code/cloudera_challenge/figures/outlier_CA_San_Mateo_County_vs_rest.png}
    \caption{\small Cost to perform different procedures in CA - San Mateo County vs other regions}
    \label{fig:outlier_CA_San_Mateo_County_vs_rest}
\end{figure*} 



%---------------------
\section{Anomalous Patients}
\label{sec:anom_patients}
%---------------------
This part of the analyis was once again rather open-ended in terms of the questions asked, but the overall goal was to identify a set of 10,000 patients who are considered most unusal, based on a set of 50,045 patients who have been labeled ``unusal'' by the Medicare staff. We first tried to understand what made these 50,045 patients stand out from the remaining. We were given data on 100 million patients as an XML file, containing the ID, age group, income group and gender of the patients; additionally, we were given a set of 300 million combinations (as a set of 12 ADT files) of patients and procedures which described which patients underwent which procedures and when. We wrote a custom XML parser in Java (ConvertXmlToCsv.java) to convert the XML file on patients to a CSV file, and converted the 12 ADT files on patients and procedures to CSV file with a simple use of the Unix tr command \cite{utr}. Next, we wrote a custom Java program (SamplePatients.java) to go through the CSV file of patients and create a sample of expected size 250,000 with the following logic: if the patient is one of the 50k labeled as anomalous, then include it in the sample; otherwise, include it in the sample with a probability of 0.002 (the probability was derived as the ratio of the expected sample size to the size of the original data file). Once this sample was obtained, we uploaded it to a PostgreSQL table with the script \texttt{create\_patient\_table.sql}. Next, we wrote another custom Java program (SampleProcedures.java) to go through the 12 CSV files on patients and procedures and pick the procedures only for the patients who have already been sampled. Note that we did not take the sample of procedures independently of the sample of patients to ensure we have enough overlap between the sampled patients and the sampled procedures for the subsequent analysis. We uploaded this to another PostgreSQL table. We then combined the sampled patients with the procedures they underwent, and created a wide matrix with patients as the rows and procedures as the columns (features).\\

In order to understand what makes these 50,045 patients unusual, we took a sample of 50,000 patients from the 200k unlabeled sampled patients, labeled them as benign, combined it with the $\sim$50k patients labeled as anomalous, and fitted a decision tree on the $\sim$100k patients thus obtained. We obtained a False Negative Rate of 2.74\%, and a False Positive Rate of 15.24\%. The main goal of doing the binary classification was to see which variables (procedures) turn out to be most important: the decision tree got built with 6 procedure variables: 0605 (Level 2 Hospital Clinic Visits), 0604 (Level 1 Hospital Clinic Visits), 0606 (Level 3 Hospital Clinic Visits), 0607 (Level 4 Hospital Clinic Visits), 0368 (Level II Pulmonary Tests) and 0690 (Level I Electronic Analysis of Devices). The relative importance of these 6 variables were (scaled to add up to 100) 50, 26, 16, 4, 3 and 1. Note that most of these procedures are different levels of hospital clinic visits. The various nodes of the decision tree showed that patients who visited hospital clinics were mostly the ones not labeled as unusual, implying that the anomalous patients are probably those who reported some other procedures but did not make enough hospital visits. This is unusual because usually people need to visit a clinic several times to consult physicians before undergoing any major procedure, like a surgery.\\

Next, we fitted a logistic regression model with a similar dataset of $\sim$100k patients with $\sim$50k anomalous and $\sim$50k benign. The model delivered a False Negative Rate of 3.6\%, and a False Positive Rate of 7.05\% - much better than the decision tree classifier. Then we applied the model on the remaining $\sim$150k patients and computed their probabilities of being anomalous. Then, we picked the 10,000 patients out of these $\sim$150k whose predicted probabilities of being anomalous were the highest. The IDs of these patients are listed in the attached file part3.csv. The first one had a probability of 0.999998, and the last one had a probaility of 0.5417 of being anomalous, so even the last one was more likely to be anomalous than not. Even the logistic regression model confirmed that the variables related to the hospital clinic visits are important ones.\\

To analyze how these 10,000 patients we label as anomalous are different from the remaining patients (by ``remaining'', we mean all patients except these 10,000 and the $\sim$50k initially provided as anomalous), we checked the percentage of patients from these two groups who have undergone various procedures. The results are listed in Figure~\ref{fig:analyze_additional_10000}. We see that the remaining patients (green bars) are much more likely to make the hospital clinic visits than the 10,000 anomalous; whereas the 10,000 anomalous are much more likely to have undergone procedures 470 (MAJOR JOINT REPLACEMENT OR REATTACHMENT OF LOWER EXTREMITY W/O MCC), 871 (SEPTICEMIA OR SEVERE SEPSIS W/O MV 96+ HOURS W MCC), 392 (ESOPHAGITIS, GASTROENT and MISC DIGEST DISORDERS W/O MCC) and 194 (SIMPLE PNEUMONIA and PLEURISY W CC). These procedure codes were all listed under the inpatient procedures in the data that we used in Section~\ref{sec:ppr}, so anomalous patients are much more likely to have reported inpatient procedures whereas the general population is much more likely to report outpatient procedures. 

\begin{figure*}[!h]
    \centering
    \includegraphics[width=18cm,height=9cm]{/Users/blahiri/healthcare/code/cloudera_challenge/figures/analyze_additional_10000.png}
    \caption{\small Percentage of patients from 10,000 anomalous vs remaining undergoing various procedures. Green bars are much higher than red ones for procedures 0604, 0605, 0606, 0607; whereas red bars are much higher than green bars for procedures 470 and 871.}
    \label{fig:analyze_additional_10000}
\end{figure*} 


\bibliographystyle{abbrv}
\bibliography{health}  
\end{document}
