%---------------------
\section{Anomalous Patients}
\label{sec:anom_patients}
%---------------------
This part of the analyis was once again rather open-ended in terms of the questions asked, but the overall goal was to identify a set of 10,000 patients who are considered most unusal, based on a set of 50,045 patients who have been labeled ``unusal'' by the Medicare staff. We first tried to understand what made these 50,045 patients stand out from the remaining. We were given data on 100 million patients as an XML file, containing the ID, age group, income group and gender of the patients; additionally, we were given a set of 300 million combinations (as a set of 12 ADT files) of patients and procedures which described which patients underwent which procedures and when. We wrote a custom XML parser in Java (ConvertXmlToCsv.java) to convert the XML file on patients to a CSV file, and converted the 12 ADT files on patients and procedures to CSV file with a simple use of the Unix tr command \cite{utr}. Next, we wrote a custom Java program (SamplePatients.java) to go through the CSV file of patients and create a sample of expected size 250,000 with the following logic: if the patient is one of the 50k labeled as anomalous, then include it in the sample; otherwise, include it in the sample with a probability of 0.002 (the probability was derived as the ratio of the expected sample size to the size of the original data file). Once this sample was obtained, we uploaded it to a PostgreSQL table with the script \texttt{create\_patient\_table.sql}. Next, we wrote another custom Java program (SampleProcedures.java) to go through the 12 CSV files on patients and procedures and pick the procedures only for the patients who have already been sampled. Note that we did not take the sample of procedures independently of the sample of patients to ensure we have enough overlap between the sampled patients and the sampled procedures for the subsequent analysis. We uploaded this to another PostgreSQL table. We then combined the sampled patients with the procedures they underwent, and created a wide matrix with patients as the rows and procedures as the columns (features).\\

In order to understand what makes these 50,045 patients unusual, we took a sample of 50,000 patients from the 200k unlabeled sampled patients, labeled them as benign, combined it with the $\sim$50k patients labeled as anomalous, and fitted a decision tree on the $\sim$100k patients thus obtained. We obtained a False Negative Rate of 2.74\%, and a False Positive Rate of 15.24\%. The main goal of doing the binary classification was to see which variables (procedures) turn out to be most important: the decision tree got built with 6 procedure variables: 0605 (Level 2 Hospital Clinic Visits), 0604 (Level 1 Hospital Clinic Visits), 0606 (Level 3 Hospital Clinic Visits), 0607 (Level 4 Hospital Clinic Visits), 0368 (Level II Pulmonary Tests) and 0690 (Level I Electronic Analysis of Devices). The relative importance of these 6 variables were (scaled to add up to 100) 50, 26, 16, 4, 3 and 1. Note that most of these procedures are different levels of hospital clinic visits. The various nodes of the decision tree showed that patients who visited hospital clinics were mostly the ones not labeled as unusual, implying that the anomalous patients are probably those who reported some other procedures but did not make enough hospital visits. This is unusual because usually people need to visit a clinic several times to consult physicians before undergoing any major procedure, like a surgery.\\

Next, we fitted a logistic regression model with a similar dataset of $\sim$100k patients with $\sim$50k anomalous and $\sim$50k benign. The model delivered a False Negative Rate of 3.6\%, and a False Positive Rate of 7.05\% - much better than the decision tree classifier. Then we applied the model on the remaining $\sim$150k patients and computed their probabilities of being anomalous. Then, we picked the 10,000 patients out of these $\sim$150k whose predicted probabilities of being anomalous were the highest. The IDs of these patients are listed in the attached file part3.csv. The first one had a probability of 0.999998, and the last one had a probaility of 0.5417 of being anomalous, so even the last one was more likely to be anomalous than not. Even the logistic regression model confirmed that the variables related to the hospital clinic visits are important ones.\\

To analyze how these 10,000 patients we label as anomalous are different from the remaining patients (by ``remaining'', we mean all patients except these 10,000 and the $\sim$50k initially provided as anomalous), we checked the percentage of patients from these two groups who have undergone various procedures. The results are listed in Figure~\ref{fig:analyze_additional_10000}. We see that the remaining patients (green bars) are much more likely to make the hospital clinic visits than the 10,000 anomalous; whereas the 10,000 anomalous are much more likely to have undergone procedures 470 (MAJOR JOINT REPLACEMENT OR REATTACHMENT OF LOWER EXTREMITY W/O MCC), 871 (SEPTICEMIA OR SEVERE SEPSIS W/O MV 96+ HOURS W MCC), 392 (ESOPHAGITIS, GASTROENT and MISC DIGEST DISORDERS W/O MCC) and 194 (SIMPLE PNEUMONIA and PLEURISY W CC). These procedure codes were all listed under the inpatient procedures in the data that we used in Section~\ref{sec:ppr}, so anomalous patients are much more likely to have reported inpatient procedures whereas the general population is much more likely to report outpatient procedures. 

\begin{figure*}[!h]
    \centering
    \includegraphics[width=18cm,height=9cm]{/Users/blahiri/healthcare/code/cloudera_challenge/figures/analyze_additional_10000.png}
    \caption{\small Percentage of patients from 10,000 anomalous vs remaining undergoing various procedures. Green bars are much higher than red ones for procedures 0604, 0605, 0606, 0607; whereas red bars are much higher than green bars for procedures 470 and 871.}
    \label{fig:analyze_additional_10000}
\end{figure*} 
