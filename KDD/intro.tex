%---------------------
\section{Introduction}
\label{sec:introduction}
%---------------------
Healthcare expenditure in the US is growing concern. The total annual healthcare expenditure in 2012 was \$2.8 trillion, the prescription drug spending accounting for \$260.8 billion \cite{healthaffairs}. As the percentage of GDP (17.9), it is the highest among all nations. The healthcare cost per capita in 2010 was \$8,233, which was one-fifth of the personal income per capita (\$42,693). The health expenditure per capita was about \$3,000 more than Norway, Switzerland or Netherlands, nations who came closest \cite{countrycomp} in terms of healthcare cost per capita. However, the huge expenditure does not necessarily buy Americans a better healthcare: in 2010, the number of practicing physicians per 1,000 people was only 2.4: whereas the corresponding number for the OECD countries was 3.1; there were 2.6 hospital beds per 1,000 people in 2009, whereas the OECD average was 3.4. In terms of life expectancy, the US ranked $28^{th}$ among the OECD countries in 2008, and it ranked below the average \cite{lifeex}. Moreover, a lot of the huge expense in the US is being attributed to unnecessary and inefficient measures, which include, but are not limited to: redundant medical tests \cite{seattle}, high cost of patented prescription drugs, infections caught from hospitals and frequent readmissions to hospitals. In 2011, the unnecessary expenditures added up to \$476 billion (18\%) to \$992 billion (37\%) of total (2.6 trillion) \cite{unnecessary}.\\

Policymakers and practitioners have started taking initiatives to reduce the overall expenditure and to tame its growth. Some of the measures suggested (and taken) are to increase the use of generic drugs as opposed to patented brand-name drugs, to identify services that were once considered good healthcare but now are suspected to lack evidence of benefit, to reduce the time spent due to administrative complexities, etc. Awareness about the usage of health data to derive useful insights for the patients, as well as for the service providers, is also on rise. As part of this initiative, the    
Department of Health and Human Services (HHS) took the Health Data Initiative \cite{hdi}, as a result of which, many health-related datasets, some of which are available from the insurance providers, are now publicly available on sites like \cite{healthdata}.\\

We attempt to address the problem of rise in healthcare expenditure by asking the following question: what factors increase an individual's expenses on healthcare? Is it because people develop certain chronic conditions? Is it because of age? Is it due to side-effects of drugs they have been prescribed? More specifically, given an individual's demographic and medical information, how well can we predict whether or not the individual's healthcare expense will rise next year? Although we formulated the problem as a binary classification problem, and experimented with various machine learning techniques to improve the classification accuracy, our goal was to go beyond simply achieving a high predictive accuracy: to identify how important the different factors are in increasing a person's healthcare expenditure.\\

Besides helping an individual, this can also be of interest to the patients/beneficiaries, the insurance providers and the healthcare service providers in ways we explain below:
\begin{itemize}
\item {\bf Beneficiaries:} If the beneficiaries know in advance that they are under high risk of an increase in expenditure for the next year, they can choose the insurance plans with higher deductible with more confidence. That way, although the annual out-of-pocket expense until the deductible is met would be higher, the beneficiary can benefit from a low monthly premium, and has to pay only 20-30\% co-insurance out-of-pocket once the deductible is met. 
\item {\bf Insurance providers:} The insurance providers can project the cost and revenue for the next year more accurately, given the data about the beneficiaries registered with them.
\item {\bf Healthcare service providers:} Hospital admissions and stays are intrinsically expensive. If doctors and hospitals know which patients are the high-risk ones, they can avoid readmission by taking preventive measures, e.g., arranging more frequent check-ups as outpatients, making the patients wear inexpensive sensors when they are discharged to monitor their conditions, etc.
\end{itemize}

{\bf Dataset: } We worked on the dataset ``CMS 2008-2010 Data Entrepreneurs$^\prime$ Synthetic Public Use File (DE-SynPUF)'' available at \cite{desynpuf}. This is an anonymized dataset obtained from Medicare \cite{medicare}, and has data about 6.87 million beneficiaries, their inpatient claims (claims filed on hospitalization), outpatient claims and prescription drug events. Since inpatient claims often form a significant fraction of an individual's annual healthcare expenses, we analyzed the expenditure on the inpatient claims. However, we use the other datasets (as well as some auxiliary datasets) to derive features on which prescribed drugs the beneficiaries took, which medical conditions they got diagnosed with as inpatients or even as outpatients, and what chronic conditions they had. One problem of healthcare datasets, in general, is that different aspects of a person's health are often siloed in different databases. This dataset is free from that problem, and that was one of the main reasons we chose it. 
The data is anonymized to protect the privacy of the beneficiaries, but the various parts of the dataset are linked through beneficiary IDs that were created by applying a hash function on the original beneficiary IDs. We present more details of the dataset in Section~\ref{sec:data}.

{\bf Contributions:} Our contribution can be summarized as follows:
\begin{itemize}
\item We used a large, publicly available dataset with 6.87 million beneficiaries and their comprehensive health history for a span of three years (2008-2010) to investigate what factors led to increase in expenditure as inpatient between 2008 and 2009. We formulated it as a binary classification problem. However, since the positive class (beneficiaries whose inpatient expenditure increased between 2008 and 2009) constituted only 14\% of the original dataset we started with, applying binary classification algorithms on that data itself (or a simple random sample of it) would have resulted in high false negative rate (low sensitivity), even if it had achieved high overall accuracy - the so-called ``class imbalance'' problem \cite{FP00}. To avoid this, we oversampled the minority class and undersampled the majority class to create a sample with equal representation of both classes.  
\item We experimented with various classification algorithms and finally chose six (SVM \cite{CV95}, gradient boosting machine \cite{Friedman01}, conditional inference tree \cite{HHZ06}, logistic regression, naive Bayes and neural networks \cite{Ripley96}) that performed best on a held-out test data set, and finally applied stacked generalization \cite{Wolpert92} as the ensemble technique. This achieved a sensitivity (recall) of 80\%, an overall accuracy of 77.56\% and a precision of 76.46\% on the test data set. 
\end{itemize}

 
