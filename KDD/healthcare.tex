\documentclass{sig-alternate}


\usepackage{amsmath}
\usepackage{booktabs}
\usepackage{pgfplots}
\usepackage{pgfplotstable}

\newtheorem{lemma}{Lemma}[section]
\newtheorem{fact}{Fact}[section]
\newtheorem{theorem}{Theorem}[section]
\newtheorem{question}{Question}[section]
\newtheorem{constraint}{Constraint}[section]
\newtheorem{definition}{Definition}[section]
\newtheorem{problem}{Problem}
\newcommand{\remove}[1]{}

\usepackage{graphicx,epsfig,subfig,float}
\usepackage[algoruled,linesnumbered]{algorithm2e}

\newcommand{\qedsymb}{\hfill{\rule{2mm}{2mm}}}
%\newenvironment{proof}{\begin{trivlist}
%\item[\hspace{\labelsep}{\bf\noindent Proof: }]}{\qedsymb\\\end{trivlist}}

\begin{document}

\title{Predicting Expenditure Increase in Healthcare from Medicare Data}

\numberofauthors{2}
\author{
\alignauthor
Bibudh Lahiri\\
       \affaddr{Impetus Technologies}\\
       \affaddr{720 University Avenue}\\
       \affaddr{Los Gatos, CA 95032}\\
       \email{blahiri@impetus.com}
% 2nd. author
\alignauthor
Nitin Agarwal\\
       \affaddr{Impetus Infotech (India)}\\
       \affaddr{18 Palasia, A.B. Road}\\
       \affaddr{Indore, India, 452001}\\
       \email{nitin.agarwal@impetus.co.in}
}

\maketitle
\begin{abstract}
Healthcare expenditure is a growing concern in the US. In 2012, the total US annual healthcare expenditure reached \$2.8 trillion. As a percentage of GDP, it is the highest among all the nations worldwide. In this study we investigate the variability of patient healthcare expenses year-on-year, depending on the different medical conditions patients get diagnosed with, the prescription drugs they consume and the demographic variables. We work with anonymized but publicly available Medicare data, which has more than 114,000 beneficiaries and more than 12,400 features. We address the problem of accurately predicting which beneficiaries' inpatient claim amounts increased between 2008 and 2009, using an ensemble of seven different classification algorithms. We achieved a sensitivity (recall) of 80\%, an overall accuracy of 77.56\% and a precision of 76.46\% on the test data set. We demonstrate its benefits to the healthcare stakeholders: the insurance providers for projecting cost and revenue more accurately, the high-risk patients for choosing the right insurance plan, and the healthcare providers for deciding which patients need additional monitoring. Our research shows that kidney conditions, COPD (chronic obstructive pulmonary disease), hypertension, stroke/transient ischemic attack, cancer and osteoporosis are among the most influential conditions behind expenditure increase.  
\end{abstract}

\category{H.2.8}{DATABASE MANAGEMENT}{Database Applications}[Data Mining]
\category{I.2.6}{ARTIFICIAL INTELLIGENCE}{Learning}[Concept learning]

\terms{Algorithms, Experimentation, Management}

\keywords{Healthcare, expenditure, binary classification}

%---------------------
\section{Introduction}
\label{sec:introduction}
%---------------------
Healthcare expenditure in the US is growing concern. The total annual healthcare expenditure in 2012 was \$2.8 trillion, the prescription drug spending accounting for \$260.8 billion \cite{healthaffairs}. As the percentage of GDP (17.9), it is the highest among all nations. The healthcare cost per capita in 2010 was \$8,233, which was one-fifth of the personal income per capita (\$42,693). The health expenditure per capita was about \$3,000 more than Norway, Switzerland or Netherlands, nations who came closest \cite{countrycomp} in terms of healthcare cost per capita. However, the huge expenditure does not necessarily buy Americans a better healthcare: in 2010, the number of practicing physicians per 1,000 people was only 2.4: whereas the corresponding number for the OECD countries was 3.1; there were 2.6 hospital beds per 1,000 people in 2009, whereas the OECD average was 3.4. In terms of life expectancy, the US ranked $28^{th}$ among the OECD countries in 2008, and it ranked below the average \cite{lifeex}. Moreover, a lot of the huge expense in the US is being attributed to unnecessary and inefficient measures, which include, but are not limited to: redundant medical tests \cite{seattle}, high cost of patented prescription drugs, infections caught from hospitals and frequent readmissions to hospitals. In 2011, the unnecessary expenditures added up to \$476 billion (18\%) to \$992 billion (37\%) of total (2.6 trillion) \cite{unnecessary}.\\

Policymakers and practitioners have started taking initiatives to reduce the overall expenditure and to tame its growth. Some of the measures suggested (and taken) are to increase the use of generic drugs as opposed to patented brand-name drugs, to identify services that were once considered good healthcare but now are suspected to lack evidence of benefit, to reduce the time spent due to administrative complexities, etc. Awareness about the usage of health data to derive useful insights for the patients, as well as for the service providers, is also on rise. As part of this initiative, the    
Department of Health and Human Services (HHS) took the Health Data Initiative \cite{hdi}, as a result of which, many health-related datasets, some of which are available from the insurance providers, are now publicly available on sites like \cite{healthdata}.\\

We attempt to address the problem of rise in healthcare expenditure by asking the following question: what factors increase an individual's expenses on healthcare? Is it because people develop certain chronic conditions? Is it because of age? Is it due to side-effects of drugs they have been prescribed? More specifically, given an individual's demographic and medical information, how well can we predict whether or not the individual's healthcare expense will rise next year? Although we formulated the problem as a binary classification problem, and experimented with various machine learning techniques to improve the classification accuracy, our goal was to go beyond simply achieving a high predictive accuracy: to identify how important the different factors are in increasing a person's healthcare expenditure.\\

Besides helping an individual, this can also be of interest to the patients/beneficiaries, the insurance providers and the healthcare service providers in ways we explain below:
\begin{itemize}
\item {\bf Beneficiaries:} If the beneficiaries know in advance that they are under high risk of an increase in expenditure for the next year, they can choose the insurance plans with higher deductible with more confidence. That way, although the annual out-of-pocket expense until the deductible is met would be higher, the beneficiary can benefit from a low monthly premium, and has to pay only 20-30\% co-insurance out-of-pocket once the deductible is met. 
\item {\bf Insurance providers:} The insurance providers can project the cost and revenue for the next year more accurately, given the data about the beneficiaries registered with them.
\item {\bf Healthcare service providers:} Hospital admissions and stays are intrinsically expensive. If doctors and hospitals know which patients are the high-risk ones, they can avoid readmission by taking preventive measures, e.g., arranging more frequent check-ups as outpatients, making the patients wear inexpensive sensors when they are discharged to monitor their conditions, etc.
\end{itemize}

{\bf Dataset: } We worked on the dataset ``CMS 2008-2010 Data Entrepreneurs$^\prime$ Synthetic Public Use File (DE-SynPUF)'' available at \cite{desynpuf}. This is an anonymized dataset obtained from Medicare \cite{medicare}, and has data about 6.87 million beneficiaries, their inpatient claims (claims filed on hospitalization), outpatient claims and prescription drug events. Since inpatient claims often form a significant fraction of an individual's annual healthcare expenses, we analyzed the expenditure on the inpatient claims. However, we use the other datasets (as well as some auxiliary datasets) to derive features on which prescribed drugs the beneficiaries took, which medical conditions they got diagnosed with as inpatients or even as outpatients, and what chronic conditions they had. One problem of healthcare datasets, in general, is that different aspects of a person's health are often siloed in different databases. This dataset is free from that problem, and that was one of the main reasons we chose it. 
The data is anonymized to protect the privacy of the beneficiaries, but the various parts of the dataset are linked through beneficiary IDs that were created by applying a hash function on the original beneficiary IDs. We present more details of the dataset in Section~\ref{sec:data}.

{\bf Contributions:} Our contribution can be summarized as follows:
\begin{itemize}
\item We used a large, publicly available dataset with 6.87 million beneficiaries and their comprehensive health history for a span of three years (2008-2010) to investigate what factors led to increase in expenditure as inpatient between 2008 and 2009. We formulated it as a binary classification problem. However, since the positive class (beneficiaries whose inpatient expenditure increased between 2008 and 2009) constituted only 14\% of the original dataset we started with, applying binary classification algorithms on that data itself (or a simple random sample of it) would have resulted in high false negative rate (low sensitivity), even if it had achieved high overall accuracy - the so-called ``class imbalance'' problem \cite{FP00}. To avoid this, we oversampled the minority class and undersampled the majority class to create a sample with equal representation of both classes.  
\item We experimented with various classification algorithms and finally chose six (SVM \cite{CV95}, gradient boosting machine \cite{Friedman01}, conditional inference tree \cite{HHZ06}, logistic regression, naive Bayes and neural networks \cite{Ripley96}) that performed best on a held-out test data set, and finally applied stacked generalization \cite{Wolpert92} as the ensemble technique. This achieved a sensitivity (recall) of 80\%, an overall accuracy of 77.56\% and a precision of 76.46\% on the test data set. 
\end{itemize}

 

%---------------------
\section{Details on Data}
\label{sec:data}
%---------------------
As we mentioned in Section~\ref{sec:introduction}, the DE-SynPUF dataset \cite{desynpuf} has information about 6.87 million beneficiaries, their inpatient claims (claims filed on hospitalization), outpatient claims and prescription drug events. Each of these components is available in 20 different partitions, and we worked with the first partition (about 5\% of the data). We created a PostgreSQL \cite{postgresql} database with this partition, and used \texttt{RPostgreSQL} \cite{rpostgresql} to connect to the database. The volumes of the different parts of the dataset are in Table~\ref{tab:desynpuf_data}.\\

\begin{table}[ht]
\caption{Volume of DE-SynPUF dataset}
\begin{tabular}{lrr}
\hline
Entities & Whole data & Subset explored\\
\hline
Beneficiaries & 6.87 million & 116k\\
Inpatient claims & 1.3 million & 66.7k\\
Outpatient claims & 15.8 million & 790.8k\\
Prescription drug events & 111 million & 5.5 million\\
\hline
\end{tabular}
\label{tab:desynpuf_data}
\end{table}

We now present some details of each of the components of the dataset (a mode detailed data dictionary is availble at \cite{datadic}): 
\begin{enumerate}
\item {\bf Beneficiaries:} We worked with a subset of 114,538 beneficiaries who were registered in both 2008 and 2009 with Medicare. The beneficiary subset provides the gender, date of birth, gender, race, whether the beneficiary had end stage renal disease, whether the beneficiary had any of the following chronic conditions: Alzheimer or related disorders or senility, heart failure, kidney disease, cancer, chronic obstructive pulmonary disease (COPD), depression, diabetes, ischemic heart disease, osteoporosis, rheumatoid arthritis and osteoarthritis, stroke/transient ischemic attack. The beneficiary summary is stored at a per-year level. That helped us to derive features like whether a beneficiary developed a chronic condition in 2009 which she did not have in 2008, and as we will show later, those derived features turned out to be pretty strong factors behind cost increase. The beneficiary dataset also had information about the total amount that Medicare reimbursed in a year for inpatient and outpatient claims by a beneficiary.  
\item {\bf Inpatient claims:} We worked with a subset of 66.7k inpatient claims, which had, for each claim, the admission and discharge dates for hospitalization episodes, and a list of three types of codes:
\begin{enumerate}
\item {\bf Diagnosis codes: }These are ICD9 \cite{icd9} codes for beneficiary's principal or other diagnosis. They capture the physician's opinion of the patient's specific illnesses, signs, symptoms, and complaints. For example, 414.00 is the ICD9 code for Coronary Atherosclerosis.
\item {\bf Procedure codes: }These are ICD9 codes for specific health interventions made by medical professionals, e.g., 4513 stands for ``Other endoscopy of small intestine''.
\item {\bf HCPCS codes: }These are CPT codes \cite{cpt} for tasks and services a medical practitioner may provide to a Medicare patient including medical, surgical and diagnostic services, e.g., 90658 stands for ``flu shot''.
\end{enumerate}
\item {\bf Outpatient claims:} We worked with a subset of 791k outpatient claims, which had, for each claim, a list of diagnosis codes, procedure codes and HCPCS codes, like inpatient claims.
\item {\bf Prescription Drug Events: }We worked with a subset of 5.5 million prescription drug events, where each record had a product service ID, which identifies the dispensed drug using a National Drug Code (NDC); the number of units, grams or milliliters dispensed and the number of days' supply of medication.
\end{enumerate}

We complemented this dataset by auxiliary ones that give: 
\begin{itemize}
\item The short and long descriptions of diagnoses for given ICD9 codes
\item The short and long descriptions of procedures for given ICD9 codes
\item A dataset that, given the NDC code of a drug, gives the name of the substance (the chemical ingredient), the product type (``human OTC drug'', ``human prescription drug'' etc), the non-proprietary name, routename (intramuscular, intravenous, oral etc), labeler name etc. For example, NDC code ``0003-0855'' stands for a drug named Sprycel \cite{sprycel}, that is used for treating a type of leukemia (CML), has to be taken orally, has a non-proprietary name Dasatinib, which also is the name of the substance behind. As we will discuss later, we actually used the substance names instead of the NDC codes when we did the modeling, since many drugs with different NDC codes and different proprietary names can have the same substance behind.  
\end{itemize}
 
Before we started working on the classification problem, we did some exploratory analysis of the DE-SynPUF dataset, and we present some of its results now: we had a total of 114,538 beneficiaries who were registered in both 2008 and 2009 with Medicare. 55\% of them were female, 45\% male. The median age at the start of 2009 was 72. We show the distribution of age (histogram and kernel density estimate) at the beginning of 2009 in Figure~\ref{fig:age_2009}. We notice a spike at the age 65, since Medicare is offered to people with age 65 or more, and with younger people with disabilities or end stage renal disease (ESRD). The median number of inpatient claims, among people who did get hospitalized, was one, for both 2008 and 2009. A vast majority of inpatient claim amounts were 0, implying most of the registered beneficiaries never got hospitalized. However, some non-zero values of inpatient claim amounts were very large. In Figures~\ref{fig:ip_claim_amt_2008_distn} and ~\ref{fig:ip_claim_amt_2009_distn}, we show the distribution of inpatient claim amounts for 2008 and 2009, respectively. We see that although 75-80\% of the patients had an inpatient claim amounts between \$0 and \$100, the fraction of patients with inpatient claim amounts \$5,000 and above is not very small. The mean amounts of inpatient claims in these two years are \$2,583 and \$2,526 respectively. Overall, we see an increase in inpatient claim amounts for 16,248 (14.2\%) beneficiaries, while for the remaining 98,290 (85.8\%) beneficiaries, it remained same or did not increase.


\begin{figure}[!h]
    \centering
    \includegraphics[width=0.4\textwidth]{/Users/blahiri/healthcare/code/figures/age_2009_distn.png}
    \caption{\small Age distribution at the beginning of 2009}
    \label{fig:age_2009}
\end{figure}

\begin{figure}[!h]
\centering
\subfloat[Inpatient claim amounts in 2008]{\includegraphics[width=0.4\textwidth]{/Users/blahiri/healthcare/code/figures/ip_claim_amt_2008_distn.png} \label{fig:ip_claim_amt_2008_distn}}\\
\subfloat[Inpatient claim amounts in 2009]{\includegraphics[width=0.4\textwidth]{/Users/blahiri/healthcare/code/figures/ip_claim_amt_2009_distn.png} \label{fig:ip_claim_amt_2009_distn}}
\caption{\small Inpatient claim amounts, showing the overall pattern did not change much between 2008 and 2009}
\end{figure}

%---------------------
\section{Towards the Model}
\label{sec:towardsmodel}
%---------------------
As the discussion in Section~\ref{sec:data} shows, the DE-SynPUF data presented to us a wealth of information, so deciding which features to use took extensive experimentation. The features can be divided into the following five logical groups:
\begin{enumerate}
\item {\bf Demographic: } Basic demographic variables like age at the beginning of 2009 and gender.
\item {\bf Chronic conditions: }We derived a set of features based on the history of chronic information: we say a beneficiary {\em developed} a chronic condition if the beneficiary reportedly did not have that condition in 2008 but had it in 2009 (note that we mentioned in Section~\ref{sec:data} that information on chronic conditions is stored at a per-year level). This gave us 11 features.
\item {\bf Diagnosed conditions: }We added the conditions people were diagnosed with in 2008 as inpatients as well as as outpatients as features. These were stored in the database as ICD9 codes under inpatient and outpatient claims. This gave us a set of 10,635 distinct conditions. 
\item {\bf Drugs taken: }We took the substances in the prescription drugs people took in 2008. This was obtained from auxiliary data on NDC codes. We used the substance name because the same substance can have different NDC codes, depending on the labeler; e.g., 0615-7593 and 10816-102 are both valid NDC codes for the substance Minoxidil used for hair loss. This gave us 1,806 possible substances.
\item {\bf Financial: } The inpatient claim amount in 2008.
\end{enumerate}

\subsection{Feature Selection}
\label{subsec:featuresel}
The groups described above gave us a total of $2 + 11 + 10635 + 1806 + 1 = 12,455$ different features for $114,538$ beneficiaries, resulting in a matrix with $1.4$ billion cells. This matrix was highly sparse. To experiment with different classification algorithms, we computed the information gain of the features arising out of diagnosed conditions and drugs taken (since these two groups contributed the maximum number of features) and took the top 30 features. This brought down the number of features used in the models to $2 + 11 + 30 + 1 = 44$. Although information gain is a widely known metric for feature selection, we define it briefly for the sake of completeness.\\

The {\em entropy} \cite{Bishop06} of a variable $Y$ that takes $m$ possible values, with the probabilities of the respective values being $p_1,{\ldots}, p_m$, is given by 
\begin{equation}
\label{eqn:entropy}
H(Y) = \sum_{j=1}^{m}p_j\log_2{p_j}
\end{equation} 

The {\em specific conditional entropy} of the response variable $Y$, given a specific value $v$ of a predictor $X$, is given by $H(Y|X = v)$, i.e., the entropy of $Y$ among only those records in which $X$ has the value $v$. The {\em conditional entropy} of $Y$ is a weighted average of the values of the specific conditional entropy, and is given by 

\begin{equation}
\label{eqn:centropy}
H(Y|X) = \sum_{l}\Pr(X = v_l)H(Y|X = v_l)
\end{equation}
 
The {\em information gain} due to a feature variable $X$, is defined as 

\begin{equation}
\label{eqn:ig}
IG(Y, X) = H(Y) - H(Y|X)
\end{equation} 
i.e., the information gain tells us how much the uncertainty in $Y$ reduces if we know the uncertainties in $Y$ for different values of $X$.\\

We show the information gain for the top 20 features in Figure~\ref{fig:info_gain}. The names of the conditions corresponding to the ICD9 codes are in Table~\ref{tab:top_conditions}. We noticed that although the feature selection was done from the set of diagnosed conditions as well as substances in prescribed drugs, the top 30 features all come from the set of diagnosed conditions. In fact, the substance with the highest information gain (Dipyridamole, a substance that inhibits thrombus formation) ranks 146, when we rank all these features by information gain. Also, we see in Table~\ref{tab:top_conditions} that two (401.9 and 401.1) of the top six features are two kinds of hypertension, and two (V58.69 and V58.61) of the top 5 are conditions developed due to long-term (current) use of other substances, three (280.9, 285.21, 285.9) are different types of anemia: this suggests that there are perhaps groups of conditions that played cruciual roles in increasing the expenditure for beneficiaries. 

\begin{table}[ht]
\caption{Top 20 conditions in terms of information gain}
\begin{tabular}{ll}
\hline
ICD9 code & Name\\
\hline
401.9 & Unspecified essential hypertension\\
250.00 & Diabetes mellitus without mention of complication\\
V58.69 & Long-term (current) use of other medications\\
272.4 & Hyperlipidemia NEC/NOS\\
V58.61 & Long-term (current) use of anticoagulants\\
401.1 & Benign essential hypertension\\
272.0 & Pure hypercholesterolemia\\
427.31 & Atrial fibrillation\\
244.9 & Unspecified acquired hypothyroidism\\
280.9 & Iron deficiency anemia, unspecified\\
285.21 & Anemia in chronic kidney disease\\
588.81 & Secondary hyperparathyroidism (of renal origin)\\
285.9 & Anemia, unspecified\\
780.79 & Other malaise and fatigue\\
496 & Chronic airway obstruction\\
414.00 & Coronary atherosclerosis of unspecified type\\
530.81 & Esophageal reflux\\
428.0 & Congestive heart failure, unspecified\\
786.50 & Chest pain, unspecified\\
311 & Depressive disorder, not elsewhere classified\\
\hline
\end{tabular}
\label{tab:top_conditions}
\end{table}

\begin{figure}[!h]
    \centering
    \includegraphics[width=0.5\textwidth]{/Users/blahiri/healthcare/code/figures/info_gain.png}
    \caption{\small Information gain for top 20 features. The ``d\_'' prefix in feature name indicates these are diagnosed conditions. We have dropped the '.' in the ICD9 codes in the plots to avoid clutter.}
    \label{fig:info_gain}
\end{figure}
 
\subsection{Sampling}
\label{subsec:sampling}
As we mentioned in Section~\ref{sec:data}, the original dataset had 14\% positive examples. We started our experiments with an $L_1$-regularized (LASSO) logistic regression algorithm \cite{glmnet} on an efficiently stored sparse matrix \cite{matrix}. We varied the regularization parameter, $\lambda$, and observed how the overall training error, cross validation error, test error, and false negative rate and false positive rate varied. Since LASSO does feature selection, the number of features selected decreased as $\lambda$ increased. When the cross-validation error hit minimum (0.1669065), the number of features selected by LASSO was only 44 (out of more than 12,400 features). However, the false negative rate was unacceptably high because of the class imbalance problem - most positive examples were getting labeled as negative by the algorithm since the training set was dominated by the negatives.\\

In order to deal with this, we took a uniform random sample of 5,000 beneficiaries (without replacement) from the 114,538 beneficiaries, and collected the 44 selected features (discussed in Subsection~\ref{subsec:featuresel}) for these 5,000 beneficiaries. Next, we split these 5,000 points into a training set of 4,000 and a test set of 1,000 points. The training set of 4,000 points had positives and negatives in almost the same ratio as the original data, and our goal was to ensure that we get 2,000 points from each class. So, we used Algorithm~\ref{algo:samplebalance} to create a balanced sample.\\

Algorithm~\ref{algo:samplebalance} is based on a simple idea: we split the input dataset, $D = \{(\mathbf{x}, y)\}_1^l$ into two disjoint, exhaustive subsets: $D_N$ is the subset with all negative labels (beneficiaries whose cost did not increase), and $D_P$ is the subset with all positive labels (beneficiaries whose cost increased). $D_P$ occupied much less than half of the dataset $D$, and $D_N$ occupied much more than half. We wanted to undersample $D_N$, and oversample $D_P$, so that the samples taken from either class has size $s = |D|/2$, so that they add up to $|D|$. We took a uniform random sample $S_N$ of size $s$ from $D_N$ without replacement. To create a sample $S_P$ of size $s = 2500$ from $D_P$ of size $16248$, we repeated each point from $D_P$ $16248 \mbox{ div } 2500 = 6$ times, and filled up the remaining $16248 - 2500{\cdot}6 = 1248$ points by randomly sampling the points from $D_P$ without replacement.


\begin{algorithm}
\vspace{-2pt}
\caption{{\sf Sampling-for-Balance}$(D)$}
\label{algo:samplebalance}

\KwIn{$D = \{(\mathbf{x}, y)\}_1^l$: $\mathbf{x}$ is the vector of the predictors and $y$ is the response variable.}
\KwOut{$D_B$, a balanced sample with equal number of positive and negative examples. The algorithm ensures $|D| = |D_B|$}

$s \gets \frac{|D|}{2}$\\
$D_N \gets \{(\mathbf{x},y) \in D|y = -1\}$\\ 
$D_P \gets \{(\mathbf{x},y) \in D|y = 1\}$\\
$S_N \gets $a uniform random sample of size $s$, taken without replacement, from $D_N$\\
$q \gets s$ div $|D_P|$\\
$r \gets s \,\bmod\, |D_P|$\\
$S_{P_1} \gets$ a multiset with each element of $D_P$ repeated $q$ times\\
$S_{P_2} \gets$ a uniform random sample of size $r$, taken without replacement, from $D_P$\\
$S_P \gets $ A multiset with all elements of $S_{P_1}$ and $S_{P_2}$\\
$D_B \gets $ A multiset with all elements of $S_P$ and $S_N$
\end{algorithm}

\begin{theorem}
The input and output datasets of Algorithm~\ref{algo:samplebalance} have the same size, i.e., $|D|= |D_B|$.
\end{theorem}
\begin{proof}
Since $D = D_N \bigcup D_P$ and $D_N \bigcap D_P = \phi$, $|D| = |D_N| + |D_P|$. By line 4 and line 1 of Algorithm~\ref{algo:samplebalance}, $|S_N| = s = |D|/2$. Also, by lines 5 and 6, $s = |D_P|q + r$. By line 7, $|S_{P_1}| = |D_P|q$. By line 8, $|S_{P_2}| = r$. By line 9, $|S_P| = |S_{P_1}| + |S_{P_2}| = |D_P|q + r = s$. By line 10, $|D_B| = |S_P| + |S_N| = s + s = |D|$.
\end{proof}
  
\remove{
\subsection{Principal Component Analysis}
We applied PCA \cite{Bishop06} on the sample of 5,000 points (before applying Algorithm~\ref{algo:samplebalance}) with 44 features. The cumulative proportions of variance explained by the 44 PCs are in Figure~\ref{fig:cpvpca}. We see that it takes the first 15 PCs to explain 50\% of the total variance. The first and the second principal components explain 12\% and 6\% of the variance, respectively. Since the cumulative proportions of variance explained increases slowly with the number of principal components, we decided to work with the 30 features directly and not with the principal components. In Figure~\ref{fig:correlation}, we show the correlations between the features and the first principal component. Looking up Table~\ref{tab:top_conditions}, we see the first PC is mostly correlated with 4019 (Unspecified essential hypertension), 25000 (Diabetes mellitus without mention of complication), 2724 (Hyperlipidemia NEC/NOS) - features that rank high in terms of information gain; and is weakly correlated with the derived features like on whether the beneficiary developed different chronic conditions (the features that start with ``dev\_'' in Figure~\ref{fig:correlation}. 

\begin{figure}[!h]
    \centering
    \includegraphics[width=0.5\textwidth]{/Users/blahiri/healthcare/code/figures/cum_prop_variance_explained.png}
    \caption{\small Cumulative Proportion of variance explained by the PCs}
    \label{fig:cpvpca}
\end{figure} 

\begin{figure}[!h]
    \centering
    \includegraphics[width=0.5\textwidth]{/Users/blahiri/healthcare/code/figures/first_pc_correlations.png}
    \caption{\small Correlations between the features and the first PC}
    \label{fig:correlation}
\end{figure}  
}

%---------------------
\section{Classification Task}
\label{sec:classification}
%---------------------
Having done the feature selection and the sample balancing as described in Section~\ref{sec:towardsmodel}, we applied various classification algorithms. Each classification algorithm 
was trained on the balanced sample of 4,000 points after running Algorithm~\ref{algo:samplebalance} on it, and tested on the held-out set of 1,000 points. In order to ensure fair comparison among the algorithms, we ensured that the split of training and test data is always the same. We present here six of the classification 
algorithms that performed best on the independent test set. We finally created an ensemble of the six algorithms through stacked generalization \cite{Wolpert92}.

\subsection{Gradient Boosting Machine}
\label{subsec:gbm}
The Gradient Boosting Machine \cite{Friedman01} needs two main ingredients for any classification or regression problem: a) a base learner, and b) a differentiable loss function that it aims to optimize eventually. The final model delivered is an {\em additive} one, it is an addition of the hypotheses that the base learner generates in $M$ different iterations, and hence its most general form is given by 
\begin{equation}
\label{eqn:additive}
F(\mathbf{x};\{{\beta}_m, \mathbf{a}_m\}_1^M) = \sum_{m=1}^M{\beta}_mh(\mathbf{x};{\mathbf{a}}_m)
\end{equation}
We used the \texttt{gbm} package of R \cite{gbm} where the base learner is a decision tree, and the differentiable loss function is the negative binomial log-likelihood function, given by 
\begin{equation}
\label{eqn:nbll}
L(y,F) = \log{(1 + \exp(-2yF))}, y \in \{-1,1\}
\end{equation}
where $F$ is half of the log-odds ratio, i.e., 
\begin{equation}
\label{eqn:hlo}
F(\mathbf{x}) = \frac{1}{2}\log{(\frac{\Pr(y = 1|\mathbf{x})}{\Pr(y = -1|\mathbf{x})})}
\end{equation}
The boosting process generates an estimate $F_m(\mathbf{x})$ of $F(\mathbf{x})$, in the $m^{th}$ ($m \in \{1,\ldots,M\}$) iteration, by the following step:
\begin{equation}
\label{eqn:boosting_step}
F_m(\mathbf{x}) = F_{m-1}(\mathbf{x}) + \sum_{j=1}^J{\gamma}_{jm}\mathbf{1}(\mathbf{x} \in R_{jm})
\end{equation}
where $\{1,\ldots,J\}$ are the indices of the terminal nodes of the decision tree created in the $m^{th}$ iteration, so $\mathbf{1}(\mathbf{x} \in R_{jm})$ is an indicator variable that indicates whether $\mathbf{x}$ belongs to the terminal node $j$ of the decision tree created in the $m^{th}$ iteration. The factor ${\gamma}_{jm}$, linked to terminal node $j$ of the decision tree created in the $m^{th}$ iteration, is the one that takes care of the eventual minimization of the loss function. Note from Equation~\ref{eqn:boosting_step}
 that $F_{m-1}(\mathbf{x})$ gets added as a component in $F_m(\mathbf{x})$, and that is how the final hypothesis $F_M(\mathbf{x})$ becomes an {\em additive} model, as claimed in Equation~\ref{eqn:additive}.\\

The base learners (decision trees) created in the different iterations differ from each other as they are fed the ``pseudo-residuals'' $\tilde{y_i}$, instead of the original responses $y_i$, where the $\tilde{y_i}$ values in the $m^{th}$ iteration are given by 
\begin{equation}
\label{eqn:pseudo_residual}
\tilde{y_i} = 2y_i/(1 + \exp(2y_iF_{m-1}({\mathbf{x}}_i))) 
\end{equation}
so the function approximation generated till the previous iteration, $F_{m-1}(\mathbf{x})$, has a role in determining the pseudo-residuals in iteration $m$.\\

The ${\gamma}_{jm}$ in Equation~\ref{eqn:boosting_step} is given by 
\begin{equation}
\label{eqn:gamma_jm}
{\gamma}_{jm} = \sum_{{\mathbf{x}}_i \in R_{jm}}\tilde{y_i}\big/\sum_{{\mathbf{x}}_i \in R_{jm}}|\tilde{y_i}|(2-|\tilde{y_i}|)
\end{equation}
For detailed proofs, one can refer to \cite{Friedman01}.\\

We applied GBM with $M = 5,000$ and included upto 2-way interactions in the trees. We actually performed grid search and cross-validation over the number of iterations and the interaction depth, with the number of iterations starting from 1000, and going up to 10000 in steps of 1000. The interaction depth was varied between 1 and 2. The results are shown in Figure~\ref{fig:gbm_cv}. We see that the CV error reduced monotonically as the number of iterations as well as the interaction depth increased. However, when we applied the models with $M = 7000$ and $M = 10000$ back on the (whole of) training and the test data, the results were as in Table~\ref{tab:gbm_cv}, which shows that although the training error and training FNR reduced as $M$ was increased from 5000 to 10000, the performance on the test set remained practically same between $M = 5000$ and $M = 7000$, and in fact slightly degraded when we moved to $M = 10000$, implying that GBM probably started overfitting beyond $M = 5000$. Also, it takes longer to train the model as the number of iterations increases, so we chose $M = 5000$ for the final model.\\

The relative influences of the covariates in GBM (scaled to add up to 100) are shown in Figure~\ref{fig:gbm_var_influence}. The relative influence of the $j^{th}$ covariate, $x_j$, is given by 
\begin{equation}
\label{eqn:rel_inf_j}
{\hat{I_j}}^2 = \frac{1}{M}\sum_{m=1}^M{{\hat{I_j}}^2(T_m)}
\end{equation}
where $\hat{I_j}(T_m)$ is the relative influence of $x_j$ in the tree generated in the $m^{th}$ iteration, $T_m$, and is given by 
\begin{equation}
\label{eqn:rel_inf_j_T}
{\hat{I_j}}^2(T) = \sum_{t=1}^{J-1}{{\hat{i_t}}^2\mathbf{1}(v_t = j)}
\end{equation}
where the summation is over the $J-1$ non-terminal nodes of the tree with $J$ terminal ones, $v_t$ is the splitting variable associated with non-terminal node $t$, and ${\hat{i_t}}^2$ is the improvement in error as a result of the split at node $t$. We see that the development of chronic conditions (variables whose names start with ``dev\_'' in Figure~\ref{fig:gbm_var_influence}) like kidney problems, COPD (chronic obstructive pulmonary disease), stroke/transient ischemic attack, cancer, osteoporosis and diabetes are among the most influential conditions behind expenditure increase. Among the diagnosed conditions (variables whose names start with ``d\_''), 4019 (Unspecified essential hypertension) and 25000 (Diabetes mellitus without mention of complication) are the most influential ones. Note that the latter two ranked the highest in terms of information gain too, as shown in Table~\ref{tab:top_conditions}. Also, figure~\ref{fig:gbm_var_influence} shows that demographic variables (bene\_sex\_ident\_cd and age\_year2), although used in the model, did not have much of an influence. 

\begin{figure}[!h]
    \centering
    \includegraphics[width=0.5\textwidth]{/Users/blahiri/healthcare/code/figures/gbm_cv.png}
    \caption{\small CV error with grid search for hyperparameters of GBM}
    \label{fig:gbm_cv}
\end{figure}

\begin{table*}[!h]
\centering
\caption{Results of grid search for GBM}
\begin{tabular}{rrrrrrrrr}
\hline
Iterations($M$) & Interaction depth & Training error & Training FNR & Training FPR & CV error & Test error & Test FNR & Test FPR\\
\hline
5000 & 2 & 0.26375 & 0.303 & 0.2245 & 0.2657311 & 0.241 & 0.2867647 & 0.233796\\
7000 & 2 & 0.249 & 0.273 & 0.225 & 0.256968 & 0.246 & 0.2867647 & 0.239583\\
10000 & 2 & 0.24025 & 0.25 & 0.2305 & 0.251982 & 0.252 & 0.294117 & 0.24537\\
\hline
\end{tabular}
\label{tab:gbm_cv}
\end{table*}

\begin{figure}[!h]
    \centering
    \includegraphics[width=0.5\textwidth]{/Users/blahiri/healthcare/code/figures/gbm_var_influence.png}
    \caption{\small Relative influences of covariates in GBM}
    \label{fig:gbm_var_influence}
\end{figure}

\subsection{Conditional Inference Trees}
Conditional Inference Trees \cite{HHZ06} combine the idea of recursive partitioning with statistical significance tests to generate decision trees, where the split variables at the non-terminal nodes are chosen by measuring the degree of association between the predictors and the response variables by the test-statistic. The chi-square test of independence \cite{chi} is used for measuring the degree of association. Because there are multiple null hypotheses to be tested (arising from the multiple covariates) while choosing the split variable at a node, the chance of rejecting some null hypothesis, when it is true, is good enough. This problem is known as the Multiple Comparisons Problem, and is dealt with by applying Bonferroni correction to adjust the $p$-values (known as multiplicity-adjustment).\\

We used the \texttt{ctree()} function of the \texttt{party} package \cite{party} in R, with the quadratic form of a linear test statistic $t$, given by \cite{HHZ06}
\begin{equation}
\label{eqn:quad_form}
c(\mathbf{t}, \mu, \Sigma) = (\mathbf{t} - \mu)^{T}{\Sigma}^+(\mathbf{t} - \mu)
\end{equation}
where 
\begin{itemize}
\item $t$ is a vector that is a {\em linear statistic}, i.e., a linear function of the values of a given covariate $x_j$, and all the values of the response variable $y$
\item $\mu$ is the conditional expectation of $t$, given the symmetric group of all permutations of the elements of $(1,\ldots,n)$, where $n$ is the number of sample observations
\item $\Sigma$ is the conditional variance of $t$, given the symmetric group of all permutations of the elements of $(1,\ldots,n)$. ${\Sigma}^+$ is the Moore-Penrose inverse of $\Sigma$.
\end{itemize}

The reason behind choosing the quadratic form in Equation~\ref{eqn:quad_form} of the linear statistic $t$ is that it follows an asymptotic ${\chi}^2$ distribution with degrees of freedom given by the rank of $\Sigma$ \cite{Rasch1995}, which makes efficient computation of the asymptotic $p$-values possible.

The conditional inference trees can be extended to create a forest of such trees, and a measure of variable importance (mean decrease in accuracy) can be obtained from such a forest. When we applied that to our data, we obtained the plot in Figure~\ref{fig:ciforest_var_influence}. Although the scales on the Y-axes between this and Figure~\ref{fig:gbm_var_influence} are different,  note the similarity between the ordering of the covariates on the X-axis: we present this as a verification of what one method considers important is also confirmed important by another method.\\

The performance of the algorithm on the training and the test datasets are listed in Table~\ref{tab:all_algos}. We see that the performance of this algorithm on the training and the test datasets are pretty comparable, implying this algorithm did not overfit. 

\begin{figure}[!h]
    \centering
    \includegraphics[width=0.5\textwidth]{/Users/blahiri/healthcare/code/figures/ciforest_var_influence.png}
    \caption{\small Importance of covariates in conditional inference forest}
    \label{fig:ciforest_var_influence}
\end{figure}

\begin{table*}[!h]
\centering
\caption{Summary of performance of all classifiers}
\begin{tabular}{p{2cm}p{2cm}rrrrrrr}
\hline
Classifier & Hyper/Control parameters & Trg error & Trg FNR & Trg FPR & Min CV error & Test error & Test FNR & Test FPR\\
\hline
Gradient Boosting Machine & bag.fraction = 0.5, 5000 iterations, logistic loss & 0.249 & 0.273 & 0.225 & 0.2569680 & 0.246 & 0.2867647 & 0.239583\\
Conditional Inference Tree & Quadratic form test statistic, Bonferroni correction for $p$-value & 0.25775 & 0.213 & 0.3025 & & 0.299 & 0.25 & 0.30671\\
Logistic Regression & & 0.255 & 0.281 & 0.229 & & 0.249 & 0.2867647 & 0.2431\\
SVM & Linear kernel, $C = 1$ & 0.254 & 0.2745 & 0.2335 & 0.2582514 & 0.245 & 0.2867647 & 0.238426\\
Neural network & Single hidden layer with 12 units, decay = $5{\cdot}10^4$, max iterations = 200 & 0.21325 & 0.086 & 0.3405 & 0.2152645 & 0.353 & 0.2721 & 0.365741\\
Naive Bayes & & 0.26575 & 0.2465 & 0.285 & & 0.279 & 0.2720588 & 0.28\\
\hline
\end{tabular}
\label{tab:all_algos}
\end{table*}


\subsection{Neural Network}
\label{subsec:nnet}
We used a neural network with a single layer of hidden units from the \texttt{nnet} \cite{nnet} library of R. Our choice of resticting the number of hidden layers to one is influenced by the well-known result by Cybenko \cite{Cybenko92}, who showed that ``arbitray decision regions can be arbitrarily well approximated by continuous feedforward neural networks with onyl a single internal, hidden layer and any continuous sigmoid nonlinearity''. We found the optimal number of units in the hidden layer by grid search, over the range of even values between 2 and 18. The results are summarized in Table~\ref{tab:nn_cv}. Although the values in this table suggest that the result with 18 units is slightly better than that with 12 units, when we applied the model with 18 units back onto the test dataset, its performance (especially the test FNR) was worse than that of the model with 12 units (test error = 0.383, test FNR = 0.352941, test FPR = 0.3877 versus the values in Table~\ref{tab:all_algos}. Also, it takes longer to train the network as the number of units go up, so we decided to go ahead with 12 units.

\begin{table*}[!h]
\centering
\caption{Results of grid search for neural network}
\begin{tabular}{rrr}
\hline
Number of units in hidden layer & Cross-validation error & Standard deviation in CV error\\
\hline
2 & 0.2924428 & 0.07616226\\
4 & 0.2310153 & 0.01854330\\
6 & 0.2280071 & 0.02138528\\
8 & 0.2264990 & 0.01746430\\
10 & 0.2209951 & 0.02012913\\
12 & 0.2152645 & 0.02472898\\
14 & 0.2140064 & 0.01972838\\
16 & 0.2207608 & 0.02240481\\
18 & 0.2137507 & 0.01569324\\
\hline
\end{tabular}
\label{tab:nn_cv}
\end{table*}

\subsection{Support Vector Machine}
\label{subsec:svm}
We used SVM \cite{CV95} with the linear kernel. Following common notation, SVM formulates the classification as the following (primal) optimization problem: 
\begin{eqnarray}
\label{eqn:svm}
\mbox{Minimize} & \frac{1}{2}{\mathbf{w}}^T{\mathbf{w}} + C\sum_{i=1}^l{\xi}_i{\nonumber}\\ 
\mbox{subject to} & y_i({\mathbf{w}}^T{\phi}({\mathbf{x}}_i) + b) \ge 1 - {\xi}_i, i = 1,\ldots,l\\
& {\xi}_i \ge 0, i = 1,\ldots,l{\nonumber}
\end{eqnarray}
which is in practice solved by optimizing the corresponding Lagrangian dual, given by
\begin{eqnarray}
\label{eqn:svm-dual}
\mbox{Minimize} & \sum_{i=1}^{l}a_i - \frac{1}{2}\sum_{i=1}^{l}\sum_{j=1}^{l}{a_i}{a_j}{y_i}{y_j}k({\mathbf{x}}_i, {\mathbf{x}}_j){\nonumber}\\ 
\mbox{subject to} & 0 \le a_i \le C, i = 1,\ldots,l\\
& \sum_{i=1}^{l}{a_i}{y_i} = 0{\nonumber}
\end{eqnarray}
where the $a_i$ values in Equation~\ref{eqn:svm-dual} are known as the Lagrange multipliers, and $k({\mathbf{x}}_i, {\mathbf{x}}_j)$ is known as the kernel function. The RBF kernel is given by  
\begin{equation}
\label{eqn:rbf-kern}
k({\mathbf{x}}_i, {\mathbf{x}}_j) = \exp(-\gamma||{\mathbf{x}}_i - {\mathbf{x}}_j||^2), \gamma > 0{\nonumber}
\end{equation}
and the linear kernel is given by
\begin{equation}
\label{eqn:lin-kern}
k({\mathbf{x}}_i, {\mathbf{x}}_j) = {{\mathbf{x}}_i}^T{\mathbf{x}}_j{\nonumber}
\end{equation}

 We chose the regularization parameter $C$ in Equation~\ref{eqn:svm} through grid search, from among the values $\{0.1, 1, 5\}$. The performance for $C = 1$ is given in Table~\ref{tab:all_algos}. We decided to go for the linear kernel after an initial experimentation with the RBF kernel, for which we performed the grid search by picking $C$ from the set $\{0.1, 1, 10, 100\}$ and $\gamma$ from $\{0.5, 1, 2, 3\}$. However, SVM hugely overfitted with the RBF kernel. As a result, although the performance with the training set was extremely good (with $\gamma$ = 3 and $C = 100$, we found the minimum CV error 0.153, training FNR = 0.024 and training FPR = 0.0135), the performance with the test set was not acceptable. This probably stemmed from reason explained below.

We know from \cite{Vapnik95} that the expected risk of a trained machine, given by
\begin{equation}
\label{eqn:risk}
R(\alpha) = \int\frac{1}{2}|y - f(\mathbf{x}, \alpha)|dP(\mathbf{x}, y)
\end{equation}
is related to the empirical risk, given by
\begin{equation}
\label{eqn:emp-risk}
R_{emp}(\alpha) = \frac{1}{2l}\sum_{i=1}^l|y_i - f({\mathbf{x}}_i, \alpha)|
\end{equation}
through the following inequality
\begin{equation}
\label{eqn:risk-relation}
R(\alpha) \le R_{emp}(\alpha) + \sqrt{\Bigg(\frac{h(\log(2l/h) + 1) - \log({\eta}/4)}{l}\Bigg)} 
\end{equation}
which holds true with probability $1 - \eta$, where $h$ is the VC dimension. So, even if $R_{emp}(\alpha)$ is small, since the VC-dimension \cite{Vapnik95} of the RBF kernel is infinite, that does not guarantee $R(\alpha)$ will be small.

\subsection{Other algorithms}
\label{subsec:other_algos}
We also applied logistic regression and naive Bayes, using the \texttt{glm} function in the \texttt{stats} \cite{stats} library, and the \texttt{naiveBayes} function in the \texttt{e1071} \cite{e1071} library of R. The performances are listed in Table~\ref{tab:all_algos}. In Figure~\ref{fig:logr_var_influence}, we plot the coefficients of the variables along with their signs from the logistic regression model, and it shows a pattern similar to Figures~\ref{fig:gbm_var_influence} and \ref{fig:ciforest_var_influence}.

\subsection{Stacking}
\label{subsec:stacking}
We used stacked generalization \cite{Wolpert92} for creating an ensemble of the classification algorithms we discussed so far. Stacking is a meta-algorithm, where, given a set $\{(\mathbf{x}, y)\}_1^l$ of sample observations, a new set $\{({\mathbf{x}}', y)\}_1^l$ is created, where ${\mathbf{x}}'$ is the set of class labels assigned to $\mathbf{x}$ by a set of classifiers, which are being included in the ensemble. Since we chose the six algorithms mentioned in Sections~\ref{subsec:gbm} to ~\ref{subsec:other_algos} to be included in the ensemble, the attributes of ${\mathbf{x}}'$ in our example were: \texttt{svm\_class},  \texttt{gbm\_class}, \texttt{citree\_class}, \texttt{lr\_class}, \texttt{nb\_class}, \texttt{nn\_class}, for labels predicted by SVM, gradient boosting machine, conditional inference tree, logistic regression, naive Bayes and neural network respectively. We took the same sample of 5,000 observations as mentioned in Section~\ref{subsec:sampling}, applied Algorithm~\ref{algo:samplebalance} to create a class-balanced sample out of it, and derived a predicted label for each of the 5,000 points through cross-validation: i.e., we split the 5,000 points into five folds, and used the data in each fold once as a validation set and 4 times as the training set, and derived the labels for the points in a fold when it was used as the validation set.\\

Once this new dataset $\{({\mathbf{x}}', y)\}_1^l$ is created, we split it randomly into two halves (so each had 2,500 points): we used one as the training set, and the other as the test set. We trained a decision tree using the \texttt{rpart} function of the \texttt{rpart} library \cite{rpart}, after performing a grid search on the parameters \texttt{minsplit} (the minimum number of observations that should be present in a node to be considered for a split) and \texttt{maxdepth} (maximum depth of any node of the final tree, with the depth of the root node treated as 0). The final contingency table out of these 2,500 test points was as follows:

\pgfplotstabletypeset[
  every head row/.style={%
    before row={\toprule 
        & \multicolumn{2}{c}{PredictedClass}\\
        \cmidrule{2-3}},
    after row=\midrule},
  every last row/.style={after row=\bottomrule},
  columns/ActualClass/.style={string type},
]
{
    ActualClass                didNotIncrease   increased   {Total by ActualClass}
    didNotIncrease                 932             310              1242
    increased                      251            1007              1258
    {Total by Predictedclass}     1183            1317              2500
}

so the recall is $\frac{1007}{1258} = 80.05\%$, the overall accuracy is $\frac{932 + 1007}{2500} = 77.56\%$, the precision is $\frac{1007}{1317} = 76.46\%$. 

\begin{figure}[!h]
    \centering
    \includegraphics[width=0.5\textwidth]{/Users/blahiri/healthcare/code/figures/logr_var_influence.png}
    \caption{\small Coefficients of covariates in logistic regression}
    \label{fig:logr_var_influence}
\end{figure}




%---------------------
\section{Conclusion}
\label{sec:concl}
%---------------------
So far, we have obtained notions of relative importance of the different features behind expenditure increase from the plots in Figures~\ref{fig:gbm_var_influence} and ~\ref{fig:ciforest_var_influence}. These were variable importance as perceived by the different classification algorithms. In Figure~\ref{fig:cond_probs}, we go back to the original dataset of 114,538 beneficiaries and plot the conditional probabilities of cost increase in presence and absence of different conditions, e.g., the 3rd bar from the left indicates that people who devloped the chronic condition stroke/transient ischemic attack (dev\_strketia) had a 39\% conditional probability of a cost increase, whereas people who did not develop this condition had only a 13\% (1/3rd) conditional probability of a cost increase. The difference between the heights of the red bars and their adjacent green bars are one way to measure how important these covariates are. We see that the conditional probability of cost increase, when these conditions were not present, hovered between 10\% and 15\%, whereas the conditional probability of cost increase, when these conditions were present, varied from 41\% to 20\% (except for gender which does not make a significant difference).\\

What figures ~\ref{fig:gbm_var_influence}, ~\ref{fig:ciforest_var_influence} and ~\ref{fig:cond_probs} all consistently point out is that the following conditions are really instrumental behind expenditure increase, and hence people with these conditions need careful monitoring to avoid costly hospitalization episodes.
\begin{enumerate}
\item {\bf dev\_strketia:} Whether the beneficiary had one or more strokes/transient ischemic attacks
\item {\bf dev\_chrnkidn:} Whether the beneficiary developed chronic kidney conditions
\item {\bf dev\_copd:} Whether the beneficiary developed COPD (chronic obstructive pulmonary disease)
\item {\bf dev\_cncr:} Whether the beneficiary developed cancer
\item {\bf d\_4019:} Whether the beneficiary got diagnosed with unspecified essential hypertension
\item {\bf d\_25000:} Whether the beneficiary got diagnosed with diabetes mellitus
\item {\bf dev\_osteoprs:} Whether the beneficiary developed osteoporosis
\end{enumerate}

\begin{figure*}[!h]
    \centering
    \includegraphics[width=0.85\textwidth]{/Users/blahiri/healthcare/code/figures/cond_probs.png}
    \caption{\small Conditional probabilities of cost increase when conditions are present or absent. Red bars indicate the conditional proabilities when conditions are present, and green bars indicate the conditional proabilities when conditions are absent. For gender (bene\_sex\_ident\_cd), red indicates male.}
    \label{fig:cond_probs}
\end{figure*}

%---------------------
\section{Future Work}
\label{sec:future}
%---------------------
As continuation of this analysis, we plan to take a deeper dive into the data: possibly with some domain expertise. Rather than splitting the beneficiaries into only two groups, we can split them into multiple groups, depending on whether the cost increase was high, medium or low; with some domain expertise, we can identify whether the conditions that people are diagnosed with can be somehow grouped, resulting in a reduction in the number of features.   


\bibliographystyle{abbrv}
\bibliography{health}  
\end{document}
